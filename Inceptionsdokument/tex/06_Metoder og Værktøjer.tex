\section{Metoder \& Værktøjer}
\subsection{Metoder i Inceptionsfacen}
\subsubsection{Brugsmønsterdiagram}
Det første skridt i udviklingen af et brugsmønster er at finde og definere de forskellige aktører der vil interagere med systemet. En aktør kan defineres som alt der kommunikerer med systemet og ikke selv er en del af systemet. Et eksempel på dette kunne være en kunde på en webshop. Disse aktører opstilles i en tabel sammen med de brugsmønstre hver aktør kan tilgå. \\

\noindent
Da kravindsamling er en evolutionær aktivitet, bliver alle aktører ikke nødvendigvis identificeret i første iteration. Det er muligt at identificere primærer aktører i løbet af første iteration, og først senere i forløbet blive i stand til at identificere sekundære aktører, når man får mere viden om systemet. \textit{Primære} aktører interagerer med systemet for at opnå påkrævede systemfunktioner, og ud fra det, få noget ud af at bruge systemet. \textit{Sekundære} aktører støtter systemet så de primærer aktører kan gøre deres arbejde. Når aktørerne er fundet kan brugsmønstrene findes. Et brugsmønster angiver et scenarie en aktør kan interagere med. \\

\noindent
Når både aktører og brugsmønstre er fundet, kan man opstille et brugsmønster-diagram for at give en visuel forståelse for hvilke aktører der kan tilgå hvilke brugsmønstre. Et eksempel på et brugsmønsterdiagram kan ses på figur \ref{fig:usecasemodel} \\

\noindent
Brugsmønstermodellen hjælper udvikleren med at forstå brugeren, så systemet kan konstrueres. Idet der er et samlet billede af hvordan systemet skal se ud, vil udviklingen være mere målfast da kravene og deres forhold til brugeren er klare.

\subsubsection{FURPS}
FURPS er en model til klassificering af ikke-funktionelle krav, og er med til at give en detaljeret beskrivelse af kravene. Akronymet står for:

\noindent
\begin{description}
    \item [Functionality:] Hvad kunden vil have. Dette inkluderer også sikkerhedsforanstaltninger.
    \item [Usability:] Hvor effektivt er produktet fra brugerens synspunkt? Er produktet æstetisk acceptabelt? Er dokumentationen fyldestgørende? 
    \item [Reliability:] Hvad er den mest acceptable system nedetid? Er systemfejl forudsigelige? Er det muligt at demonstrere hvor præcise resultaterne er? Hvordan bliver systemet gendannet?
    \item [Performance:] Hvor hurtigt skal systemet være? Hvad er den maksimale responstid? Hvad er gennemløbet? Hvor meget hukommelse bruger systemet?
    \item [Supportability:] Kan systemet testes? Er det muligt at konfigurere systemet, udvide det, installere det, og yde service på systemet.
\end{description}

\subsubsection{MoSCoW}
MoSCoW er en prioriterings model. Den bruges ofte i software udvikling.\\
Modellen i sig selv kan dog bruges, men det anbefaldes at man bruger den sammen med en \textbf{Agile proces}. \\

\noindent
MoSCoW er en vigtig model i software udvikling da, den beskriver hvilken del af softwaren der minimum skal laves før det virker. Der laves en prioritering liste med kunden om hvad de så gerne vil have først. Det bliver så stillet op i en MoSCoW model.

\noindent
\begin{description}
    \item [Must have] betyder skal have og i software udvikling betyder det, som er minimum der skal være med for at softwaren virker. 
    \item [Should have] betyder det som burde være med det kunden rigtig gerne vil have med.
    \item [Could have] betyder det som kunne være med. Hvis der er tid nok. 
    \item [Won't have (this time)] Det som der slet ikke skal prioriteres nu, men måske en anden gang.
\end{description}

%Fra Henrik: Indsæt beskrivelse af hvad der skal ske i Elaborationsfasen.
\subsection{Metoder i Elaborationsfasen}
\subsubsection{UP \& Scrum}



% Post review: rettet KanBan board -> Scrum board. Da det er det vi bruger (https://www.atlassian.com/agile/kanban/kanban-vs-scrum)
\myparagraph{UP}
'UP' er en forkortelse af 'unified process' UP består af fire faser. De fire faser er beskrevet i den rækkefølge som de eksekveres \\
\textbf{Inceptionsfasen} \\
Inceptionsfasen er der for at finde ud af om projekt overhovedet kan gennemføres, Bestemme hvilket anvendelseområde systemet har. Identificere vigtige krav og kritiske risici. \\
\textbf{Elaborationfasen} \\
I Elaborationfasen er for at lave en iterativ udvikling af de forskellige krav, design, analyse og test ud fra den overordnede kravspecifikation og den prioritering der er lavet. I Elaborationfasen vil der blive brugt Scrum. \\
\textbf{Konstruktionfasen} \\
I Konstruktionfasen vil fokus være på udvikling af komponenter og andre funktioner.
I fasen vil der blive brugt UML til at identificere, hvilke klasser og komponenter der skal være. Det er i denne fase kodeningen kommer til at ske og den første iteration af software produktet.  \\
\textbf{Overgangfasen} \\
I Overgangfasen vil der være fokus på at få et færdigt software produkt.
Det vil man gører ved at, se om man har implementeret det er aftalt og hører ens bruger om de er tilfredse med produktet

\myparagraph{Scrum} 
Scrum vil blive brugt i Elaborationsfasen, til at nedbryde de krav vi har defineret i inceptionsfasen (se krav). Der vil blive benyttet en sprint periode på 1 uge, da det liner op med det ugentlige vejledermøde. Da sprint perioden er kort (normalt bruges 1-4 uger) er det vigtigt at vi får brudt vores Epics ned til User Stories der kan nåes indenfor 1 sprint.
Vi vil i projektet bruge værktøjet ZenHub til GitHub for at integrere Scrum ind i projektet.
Dette giver os mulighed for at samle vores projekt management og kode på vores GitHub side (\url{https://github.com/creditoro}).

\noindent
Vi har til projektet et Scrum board med følgende kolonner:

\begin{table}[H]
\begin{tabular}{|p{2cm}|p{2.6cm}|p{2.3cm}|p{2.4cm}|p{3.5cm}|p{1.5cm}|}
\hline
\textbf{New Issues} & \textbf{Icebox} & \textbf{Backlog} & \textbf{In Progress} & \textbf{Done} & \textbf{Closed} \\ \hline
 & Issues med lav prioritet & Kommende issues & Igangværende issues & Færdige issues der bliver lukket næste sprint møde & \\ \hline
\end{tabular}
\caption{Scrum Board}
\label{tab:scrumboard}
\end{table} 
%End of table

\noindent
Alle issues er sorteret fra top til bund alt efter prioritet. \\


% https://www.scrumguides.org/docs/scrumguide/v2017/2017-Scrum-Guide-US.pdf#zoom=100


\noindent
I dette projektet er der blevet defineret følgende Scrum roller \\

\noindent
\begin{table}[h]
    \centering
    \begin{tabular}[h]{|p{3cm}|p{3cm}|}
        \hline
        \textbf{Rolle} & \textbf{Personer} \\
        \hline
        Scrum master & Kristian \\
        \hline
        Product owner & Jakob \\
        \hline
        Developers & Alle \\
        \hline
    \end{tabular}
    \caption{Scrum roller}
    \label{tab:Scrum_roles}
\end{table}
% role, users
% scrum master, kjako 19
% product owner, Jakob
% developers alle.



\subsubsection{Værktøjer}
I tabel \ref{tab:tools} ses de værktøjer projektgruppen har benyttet under inceptionsfasen og de værktøjer, der skal bruges fremadrettet.
\begin{longtable}{|p{4cm}|p{12cm}|}
\hline
\textbf{Værktøj} & \textbf{Beskrivelse} \\
\hline
PostgreSQL          &   PostgreSQL er en open-source objekt-relationel database server.\\
\hline
GitHub              &   GitHub er en web-baseretkollaborations platform henvendt til software udviklere, der gør det muligt at versions-kontrollere projekter.\\ 
\hline
Overleaf            &   Overleaf er en online skriveplatoform for \textbf{LaTeX}, hvor man kan være flere brugere der skriver samtidig. \\
\hline
UML                 &   Unified Modeling Language \\
\hline
IntelliJ            &   Integreret udviklings miljø, som primært bruges af gruppens medlemmer. \\
\hline
ZenHub              &   ZenHub er en platform der gør det lettere at anvende Scrum i praksis.  \\
\hline
Scrum Board         &   Et Scrum Board er et værktøj, der har til formål at gøre opgarverne i Sprint og Backlog synlige og overskuelige.\\
\hline
Pair Programming    &   Pair programming er en softwareudvilkingsteknik, hvor to programmører arbejder sammen ved én computer.\\
\hline
Klassediagram       &   Bruges til visuelt at vise hvordan softwaresystemer er opbygget. I diagrammet beskrives systemets klasser, metoder og værdier klassen indeholder, samt klassernes relationer til hinanden.\\
\hline
    \caption{Værktøjer til projektarbejdet}
    \label{tab:tools}
\end{longtable}

