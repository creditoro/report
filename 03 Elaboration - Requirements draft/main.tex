\documentclass[a4paper,12pt]{article}
% Resten af pakkerne
\usepackage[english, danish]{babel}
\usepackage{csquotes}
\usepackage{float}
\usepackage{flafter}
\usepackage{graphicx}
\usepackage{setspace}
\usepackage{enumitem}
\usepackage{multirow}
\usepackage{lmodern}
\usepackage{amssymb,amsmath}
\usepackage{ifxetex,ifluatex}
\usepackage{lastpage} % Bruges i customtitlepage til at tælle sider
\usepackage[font=small,labelfont=bf]{caption}
\usepackage{lscape}
\usepackage{xargs}
\usepackage{tabularx}
\usepackage[utf8]{inputenc}
\usepackage{xcolor}
% \usepackage{csvsimple}
\usepackage{longtable}
\usepackage{lastpage}
\usepackage{fancyhdr}

% Page numbering
\pagestyle{fancy} 
\cfoot{\thepage\ of \pageref{LastPage}}

% Margin
\usepackage{geometry}
\geometry{a4paper,  total={170mm,250mm},
 left=20mm,
 top=30mm}

% Bibliografi
\usepackage[
backend=biber,
style=alphabetic,
citestyle=authoryear
]{biblatex}
\addbibresource{appendices/bibliography.bib} %Imports bibliography file
\bibliography{appendices/bibliography.bib}

% New Commands
\newcommand\myworries[1]{\textcolor{red}{#1}} 
\newcommand{\myparagraph}[1]{\paragraph{#1}\mbox{}\\}


% Needs to be the last package included
\usepackage{hyperref}
\hypersetup{
    colorlinks=true,
    linkcolor=black,
    filecolor=magenta,      
    urlcolor=blue,
}


% Line Mellem rum
% \linespread{1.4}
\graphicspath{{./figures/}{../figures/}}

\title{Requirements draft}
\author{Group 06}
\date{March 26th 2020}

\begin{document}
\pagenumbering{gobble} % Fjerner side tal, arabic skal bruges senere 
% Sådan ser du antal sider i alt ;) 

\maketitle

\newpage
\thispagestyle{empty}
\noindent
\begin{tabular}{@{}l l} 
\textbf{Title:} & Creditoro - Functional \\ 
                & Non-functional Requirements\\
& \\
\textbf{Institution:} & Syddansk Universitet \\
& Det Tekniske Fakultet, Mærsk Mc-Kinney Møller Instituttet \\
& Campusvej 55, 5230 Odense M \\
& \\
\textbf{Uddannelse:} & Softwareteknologi \\
& \\
\textbf{Term:} & 2. term \\
& \\
\textbf{Project group:} & 06\\
& \\
\textbf{Version:} & \texttt{0.0.1-DRAFT}\\
& \\
\end{tabular}

% OwO Jakob Jakob Jakob Jakob Jakob Jakob Jakob Jakob Jakob Jakob Jakob
\vspace{-0.5mm}
\includegraphics[scale=0.07]{figures/signatures/signatureJR.jpg}
\vspace{-9.5mm}
\par\noindent\rule{\textwidth}{0.4pt}
\noindent
Jakob Rasmussen, jakra19@student.sdu.dk\\
% -end- -end- -end -end -end- -end- -end -end -end- -end- -end -end -end-

% Kenneth Kenneth Kenneth Kenneth Kenneth Kenneth Kenneth Kenneth Kenneth
\noindent
\includegraphics[scale=0.3]{figures/signatures/signature_kechr19.PNG}
\vspace{-5mm}
\par\noindent\rule{\textwidth}{0.4pt}
\noindent
Kenneth M. Christiansen, kechr19@student.sdu.dk\\
\vspace{3.5mm}
% -end- -end- -end -end -end- -end- -end -end -end- -end- -end -end -end-

% KEVIN KEVIN KEVIN KEVIN KEVIN KEVIN Kevin Kevin Kevin Kevin Kevin Kevin
\vspace{-6.5mm}
\noindent
\includegraphics[scale=0.3]{figures/signatures/signature_kepet19.png}
\vspace{-8mm}
\par\noindent\rule{\textwidth}{0.4pt}
\noindent
Kevin K. M. Petersen, kepet19@student.sdu.dk
% -end- -end- -end -end -end- -end- -end -end -end- -end- -end -end -end-

% Kristian Kristian Kristian Kristian Kristian Kristian Kristian
\noindent
\includegraphics[scale=0.04]{figures/signatures/signature_kjako19.jpg}
\vspace{-9.5mm}
\par\noindent\rule{\textwidth}{0.4pt}
\noindent
Kristian N. Jakobsen, kjako19@student.sdu.dk\\
% -end- -end- -end -end -end- -end- -end -end -end- -end- -end -end -end-

% Mathias Mathias Mathias Mathias Mathias Mathias Mathias Mathias Mathias
\noindent
\includegraphics[scale=0.120]{figures/signatures/Signatur_mara816.png}
\vspace{-5mm}
\par\noindent\rule{\textwidth}{0.4pt}
\noindent
Mathias N. Rasmussen, mara816@student.sdu.dk\\
% -end- -end- -end -end -end- -end- -end -end -end- -end- -end -end -end-

% Simon Simon Simon Simon Simon Simon Simon Simon Simon Simon Simon Simon
\noindent
\includegraphics[scale=0.042]{figures/signatures/signatureSJ.png}
\vspace{-3.5mm}
\par\noindent\rule{\textwidth}{0.4pt}
\noindent
Simon Jørgensen, sijo819@student.sdu.dk\\
% -end- -end- -end -end -end- -end- -end -end -end- -end- -end -end -end-


%Bottom of page
%\vfill

\noindent
\begin{tabular}{@{}l l}
Number of Pages:    & \pageref{LastPage} pages \\
\end{tabular}

\vspace{3.5mm}

\begin{footnotesize}
\noindent
\textbf{By signing this document every individual group member confirms that they have contributed equally to the project and is thereby liable for the content within this document. }
\end{footnotesize}
\newpage

\pagenumbering{arabic}


\section{Functional Requirements}
% note: The Rest API is responsible for validating the logic, and the client is responsible for visualizing it.
Since the REST API and the Desktop client share functionality, by having the client call the REST API to handle the logic, we have merged the two functional requirements. The desktop client will call the REST API to handle the logic, thus only minimal logic will be required in the desktop client.

\begin{longtable}{|p{1cm}|p{3.5cm}|p{11cm}|} 
    \hline
    
    % \multicolumn{3}{|c|}{}  \\ 
    % \multicolumn{3}{|c|}{Rest API}  \\ 
    % \hline
    % F01 & Authentication & The API should return a token to the caller upon successful login post request. \\
    % \hline
    % F02 & Specific credit & The API should return the credit if found when provided with an ID.  \\
    % \hline
    % F03 & List Credits & The API should return paginated credit results if the user is authenticated. \\
    % \hline
    
    %\multicolumn{3}{|c|}{}  \\ 
    \multicolumn{3}{|c|}{ \textbf{Desktop-client \& REST API}}  \\ 
    \hline
    \textbf{ID} & \textbf{Name} & \textbf{Description} \\
    \hline
    D01 & Login/out & You should be able to login and out of the system with ease by providing username and password. \\
    \hline
    D02 & Search & You should be able to search for TV-show titles as well as persons \\
    \hline
    D03 & Channel Browse & You should be able to browse channels registered in the system, and see which TV-shows are streamed on these \\
    \hline
    D04 & Display credits associated with TV-program & You should be able to display all credits associated with a TV-program \\
    \hline
    D05 & Create Channnel & The system admin should be able to create new channels. \\
    \hline
    D06 & Add system admin & The system admin should be able to add other system admins. \\
    \hline
    D07 & Add Channel admin & The system admin and channel admin should be able to add channel administrators to a channel. (Channel admin only for own channel) \\
    \hline
    D08 & Add Producer & The system admin and channel admin should be able to add producers for a show (channel admin only for own channel) \\
    \hline
    D09 & Create credit & The system admin, channel admin and producer should be able to create a credit (channel admin and producer only for own channel) \\
    \hline
    D10 & Delete credit & The system admin and channel admin should be able to delete credits (channel admin only for own channel) \\
    \hline
    D11 & Update credit & The system admin, channel admin and producer should be able to update and change credit (channel admin and producer only for own channel) \\
    \hline
    D12 & Update person & System admin, channel admin and producer should be able to update a person, mainly which shows the person has helped produce \\
    \hline
    D13 & Approve or disapprove credit & The system admin and channel admin should be able to approve or disapprove created credits before they are publicly available (channel admin only for own channel) \\
    \hline
    D14 & Create royalty user & The system admin should be able to create a royalty user \\
    \hline
    D15 & Update royalty user & The system admin and royalty user should be able to update royalty users \\% Burde det ikke være CREATE? Og hvis ikke, hvorfor har vi så ikke en create?
    \hline
    D16 & Change language & You should be able to change the language of the user interface \\
    \hline
    
    \caption{Desktop-client \& REST API: Functional Requirements}
    \label{tab:desktop_client_rest_api_funreq}
\end{longtable}

\newpage
% EPG POLLER ---------------------- EPG POLLER ------------------ EPG POLLER
\begin{table}[h]
    \begin{tabular}{|p{1cm}|p{3.5cm}|p{11cm}|}
        \hline
        %\multicolumn{3}{|c|}{}  \\
        \multicolumn{3}{|c|}{\textbf{EPG Poller}} \\
        \hline
        \textbf{ID} & \textbf{Name} & \textbf{Description} \\
        \hline
        E01 & Poll-data
        &  The EPG Poller should be able to poll data from \href{www.tvtid.tv2.dk}{tvtid.dk} \\
        \hline
        E02 & Update Database & Update the database with credits \\
        \hline
    \end{tabular}
    \caption{EPG Poller: Functional Requirements}
    \label{tab:epg_poller_funreq}
\end{table}

\section{Non-functional Requirements}

\begin{longtable}{|p{1cm}|p{3.5cm}|p{11cm}|} 
    \hline

    %\multicolumn{3}{|c|}{}  \\ 
    \multicolumn{3}{|c|}{\textbf{REST API}}  \\ 
    \hline
    \textbf{ID} & \textbf{Name} & \textbf{Description} \\
    \hline
    NR01 & Supportability &  Centralized error reporting should be available via a common interface (such as Sentry).\\
    \hline
    NR02 & Performance &  Commonly used API calls should respond within 300 milliseconds. \\
    \hline
    NR03 & Scalability &  The system should be able to handle 10K new users yearly for 25 years.\\
    \hline
    NR04 & Scalability &  The system should be able to handle 15K new credits yearly for 25 years.\\
    \hline
    NR05 & Configuration & Data persistence time should be configurable, to auto-cleanup data older than the configured value (defaults to 25 years).\\
    \hline
    NR06 & Availability &  The system should start automatically after server restarts. \\
    \hline
    NR07 & Usability &  API documentation should be available via Swagger UI. \\
    \hline
    NR08 & Security &  The server that the REST API is hosted on should only allow login via SSH. \\
    \hline
    NR09 & Security &  The server that the REST API is hosted on should only allow connections from the outside on port 443 (https), 80 (http) and 22222 (ssh). \\
    \hline
    NR10 & Installability & REST API should be deployable within a container (such as Docker). \\
    \hline
    NR11 & Configurability & REST API should be configurable using \textbf{environment} files. \\
    \hline
    NR12 & Authentication & REST API should handle authentication via token based authentication with it's clients. A token is valid for 2 hours, and is refreshed automatically after a request with the token when it's time to expire is less than an hour. \\ % When 
    \hline
    \caption{REST API: Non-functional Requirements}
    \label{tab:rest_api_nfunreq}
\end{longtable}

\begin{table}
    \centering
    \begin{tabular}{|p{1cm}|p{3.5cm}|p{11cm}|} 
        \hline
        %\multicolumn{3}{|c|}{}  \\ 
        \multicolumn{3}{|c|}{\textbf{Desktop-client}}  \\ 
        \hline
        \textbf{ID} & \textbf{Name} & \textbf{Description} \\
        \hline
        ND01 & Supportability &  Error reporting should be available via pop-out box \\
        \hline
        ND02 & Performance &  Commonly used buttons should work within 300 milliseconds \\
        \hline
        ND03 & Scalability &  The system should have a nice overview of all credits \\
        \hline
        ND04 & Scalability &  The client should be able to query search results in 5 seconds \\
        \hline
        ND05 & Configuration & Configuration should be configurable using \textbf{environment} files and GUI \\
        \hline
        ND06 & Availability &  The system should start automatically after server restarts. \\
        \hline
        ND07 & Authentication & The system should redirect to the login page when receiving a HTTP 401 (unauthorized) response from the REST API. \\
        \hline
        ND08 & Inactivity & The desktop-client should automatically detect if the authentication token is expired and redirect to login page. \\ 
        \hline
    \end{tabular}
    \caption{Desktop Client: Non-functional Requirements}
    \label{tab:desktop_client_nfunreq}
\end{table}

\begin{table}
    \begin{tabular}{|p{1cm}|p{3.5cm}|p{11cm}|} 
        \hline
        %\multirow{2}{}{\multicolumn{3}{|c|}{\textbf{EPG Poller}}} \\
        \multicolumn{3}{|c|}{\textbf{EPG Poller}} \\
        \hline
        \textbf{ID} & \textbf{Name} & \textbf{Description} \\
        \hline
        NE01 & Supportability &  Centralised error reporting should be available via a common interface (such as Sentry).\\
        \hline
        NE02 & Performance &  It should be able to poll data every hour and finish within 15 min to update the data \\
        \hline
        NE03 & Scalability &  The system should be able to poll 15K new shows every year yearly, for 25 years. \\
        \hline
        NE04 & configuration & How often the poller is run, where to poll data from and where to post data too should be configurable \\
        \hline
        NE05 & Availability &  The system should start automatically after server restarts. \\
        \hline
        NE06 & Usability & The system should warn about show's without credit \\
        \hline
        NE07 & Security & The server that the EPG poller is hosted on should only allow connections from the outside on port 22221(ssh). \\
        \hline
        NE08 & Configurability & The EPG Poller should be configurable using \textbf{environment} files. \\
        \hline
    \end{tabular}
    \caption{EPG Poller: Non-functional Requirements}
    \label{tab:epg_poller_nfunreq}
\end{table}

\newpage
\section{Acceptance Criteria}
\subsection{D01 - Login/out}
As PO I want users to be able to authenticate so that we can limit functionality to the right users.

\subsubsection{Login}
\textbf{Given} the user is not signed in.

\textbf{And} the user clicks the login button. \\
\textbf{When} the user fills out a valid username.

\textbf{And} the user fills out a valid password. 

\textbf{And} the user clicks submit (or enter hotkey) \\
\textbf{Then} a \texttt{post} request is sent to the server with the credentials over https.

\textbf{And} the user is then authenticated and a temporary token is returned to the sender.

\textbf{And} user is now signed in. \\

%\subsubsection{Logout}

\end{document}