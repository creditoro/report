\section{Procesevaluering}
Denne sektion har til formål at fremvise gruppens evaluering af udviklingsprocessen og skriveprocessen. Sektionen er delt op i to delsektioner, den ene er gruppeevalueringen og den anden er gruppemedlemmernes individuelle evalueringer. 

\subsection{Gruppeevaluering}
Nedenstående afsnit handler om, hvordan vi har evalueret processen i forhold til gruppearbejde, arbejdsformer, metoder og slutteligt hvad vi ville have gjort anderledes hvis vi skulle starte forfra med projektet.

\subsubsection{Samarbejdet internt i gruppen}
Arbejdet igennem projektforløbet har været generelt imødekommende. Samarbejdet har afspejlet det, idet at gruppen har været åbne for alle foreslag og tanker ligemeget hvilken opgave der var i fokus. Det var altid hjælp hvis man søgte det og det var ingen problemer med at afsætte tid til projektet. I de tilfælde at tid var en vanskelighed var gruppen tilbøjelig til at ændre mødetider eller sætte tiden andetstedes. Dog blev der respekteret at hvert medlem havde et liv uden for uddannelsen.\\

I starten af projektet brugte gruppen meget tid på at forventningsafstemme. Det var med til at skabe et fælles udgangspunkt for hvordan opgaven skulle gribes an, og skabte tidligt i processen et fælles mål at arbejde mod. Vi mener at det har gavnet vores projekt positivt. Efter at have dannet et fælles fundament blev gruppens Belbin roller taget i perspektiv, det viste sig dog at rollerne var vanskelige at afhænge af.

Vi fik hver især foretaget en Belbintest inden semesterstart, som vi skulle danne projektgrupper ud fra. Disse roller, mener vi, har ikke gavnet gruppearbejdet yderligere, da vi ikke føler at det er noget vi aktivt har benyttet os af. Vi mener, at teamrollerne på nuværende tidspunkt ikke er så sigende som de kunne være, da det stadig er meget nyt for os. Det, at det er vores medstuderende der har været med til at skabe vores Belbinprofil, gør også, at resultatet kan være afvigende fra hvordan man som enkelt person egentligt agere i et gruppeprojekt. Det kan skyldes at de fleste af os ikke har særlig meget erfaring med problemorienteret gruppearbejde, og derved ikke kan evaluerer hinandens roller i et projektarbejde. 

\myparagraph{Ledelse af projektet}
Hvis vi skal kigge lidt på ledelsen af projektet, så fik vi i gruppen aldrig udvalgt en bestemt leder. Dette viste sig at have en negativ effekt på vores effektivitet i løbet af sprint 3. I denne periode var tællende aktivitet i de andre fag, hvilket gjorde at vi ikke var så produktive med implementeringen af brugsmønstrene. Vi manglede her en igangsætter/leder, der kunne sørge for at de issues vi havde i vores backlog blev lukket og derved at vores deadline blev overholdt. Dette er en erfaring vi tager med til fremtidige projekter, for at undgå lignende scenarier. 

\myparagraph{Arbejdsfordelingen i projektet}
Da vi forventninsafstemmede, snakkede vi meget om hvor vores kompetencer lå, for at kunne opdele projektet bedst muligt, når nu vi valgte at lave et REST api og en EPG-Poller ud over desktop-klienten. Da en af vores gruppemedlemmer havde kompetencerne til at udvikle et REST api, fik vedkommende ansvaret for denne. Dette har gjort, at api'et primært er skrevet af en person, og at få af de andre har været med til at udvikle det. Dog har vi alle haft indflydelse på funktionaliteten, og nogle af os været inddraget i udviklingsprocessen. Det samme gør sig gældende for EPG-Polleren. Set i retrospekt kunne vi med fordel have været mere inddraget i udviklingsprocessen i de to subsystemer, fremfor at vi nu skal sætte os ind i koden. Dette var dog noget vi som gruppe aftalte fra starten, og derfor er det ikke noget vi ser som et reelt problem.\\

Da vi i projektet benyttede os af værktøjet ZenHub, var vi meget opmærksomme på at få uddelegeret de issues vi havde i vores backlog, så alle havde lignende mængde arbejde foran sig. Der er nogle af os der har skrevet flere linjer kode end andre, men dette ser vi som uundgåeligt i et projekt som dette. Det kode man har haft ansvaret for at få skrevet, er blevet gennemgået af mindst et gruppemedlem, da vi på GitHub har sat reviewers på vores pull requests som skulle godkendes, inden koden kunne blive merget med vores master branch. Dette har gjort at vi let har kunne gennemgå hinandens kode og komme med eventuelle forbedringsforslag, for at optimere desktop-klienten.

\subsubsection{Samarbejdet med vejleder}
Overordnet set mener vi, at vores samarbejde med vejleder har været fint. Vi har for det meste kunne få fyldestgørende svar på de spørgsmål vi kom med, dog mener vi at feedback har været mangelfuld til tider. Dette gjorde sig blandt andet synlig efter afleveringen af 1. iteration, hvor vi godt kunne have ønsket mere uddybende feedback fra vejleder. Dette er dog ikke noget vi på noget tidspunkt har gjort vejleder opmærksom på, så vejleder har ikke haft mulighed for at rette op på det. \\
 
At vi har været nødsaget til at holde vejledermøder over Zoom, grundet den lidt kedelige situation, føler vi også at har haft en negativ effekt på kvaliteten af de møder vi har haft. Vi ser det som værende lettere at modtage vejledning når man sidder i samme rum, sammenlignet med før vi blev hjemsendt. Dette kan skyldes at alle 45 minutter sat af til vejledermødet blev brugt på at snakke, fremfor over Zoom, hvor vi sjældent kom over 15 minutter.

\subsubsection{Projektarbejdsformen}
Projektarbejdet har generelt været fint. Gruppen følte dog at arbejdsbyrden i starten var større end i det forrige projekt. Det kom af at vi blev introduceret for UP og dens process. Efter at have sat sig ind i hvad det indebar, var det dog ligetil at finde ud af hvordan projektet skulle gribes an. 
\myworries{needs to shit}

\subsubsection{Arbejdsformer}
I skriveprocessen er der blevet anvendt to forskellige arbejdsformer, individuel skrivning og parskrivning.
De har hver haft deres fordele og ulemper, blandt andet har det været godt at skrive selv, for så skal man ikke sidde og snakke om hvad man har tænkt sig at skrive, men på den anden side kan man godt være i tvivl om hvad man skal skrive.
Her har parskrivning en af sine styrker, i at man kan sidde og diskutere hvad indholdet skulle handle om. 
Implementeringsprocessen har foregået lidt på samme måde, hvor man har lavet parprogrammering hvis der var steder man var lidt i tvivl om, ellers programmerede man individuelt.
Parprogrammeringen har været med til at skabe løsninger, der ellers ikke ville blive udviklet, da man har siddet med to forskellige syn på hvordan noget skulle programmeres.

\subsubsection{Metoder}
De metoder der er anvendt i dette projekt er brugsmønsterdiagrammer, FURPS+ og UP \& Scrum.
Vi mener at formålet med brugsmønstrene er klart, og at vi har draget nytte af dem under inceptionsfasen.
De har hjulpet under analysefasen og gjort det simpelt at holde styr på hvad hvert krav gik ud på og hvilke klasser og attributter der skulle bruges (dette blev synligt under navneordsanalyserne). 

FURPS+ blev brugt til klassificering af ikke-funktionelle krav, og var med til at give en detaljeret beskrivelse af kravene. Dette har været med til synliggøre de krav der ikke umiddelbart påvirker funktionaliteten af systemet. vi kunne dog have gjort mere brug af FURPS+

MoSCoW blev brugt til at prioritere brugsmønstrene, og var med til at hjælpe gruppen med at fokusere hvad der skulle implementeres under implementeringen. Vi kunne dog have holdt os mere til vores MoSCoW, da ikke alle af vores musts blev implementeret. Dette er en erfaring vi vil tage med os til fremtidige projekter.

UP \& Scrum er blevet brugt i sammenhæng i dette projekt, hvor vi ved hjælp af et scrumboard kunne holdte styr på udviklingen af artefakter for de enkelte faser i UP. Vi føler at vi har brugt UP efter hensigten, dog syntes vi, at der i starten af projektet var for meget fokus på analysedelen, og ikke for meget fokus på implementeringsdelen. Vi kunne godt have brugt Scrum mere seriøst. Med det menes der, at vi kunne have bedre til at udnytte de aktiviteter der finder sted i Scrum, såsom stand-up møderne og sprint review, da det kunne have hjulpet os med at undgå situationer som tidligere beskrevet, hvor der ikke blev arbejdet nok på projektet.

\subsubsection{Skriveprocessen}
Ved starten af projektet havde de fleste grupper et ønske om at rapportskrivning ville ske parralelt med udviklingen af produktet. Det kom af at medlemmerne havde erfaring med en stresset skriveprocess i forrige semester. Derfor blev der lagt vægt på at den skriftlige del af processen, blev gjort når det var muligt. Gruppen fik skrevet meget af rapporten under afleveringen af 1. iteration, dette har haft en positiv indflydelse på gruppens skriveproces op til afleveringen af det samlede projekt. En faktor der hjalp med at lette arbejdsbyrden, var en god arbejdsopdellingen, idet det var meget nemt at sætte sig på en del af opgaven og få feedback efter den var skrevet.

\subsubsection{Den tidsmæssige styring af projektet}
Til den tidsmæssige styring af projektet blev der taget udgangspunkt i den udleverede projektrammeplan. Det var ud fra denne af vi lavede vores egen tidsplan, hvor delafleveringer og andre epics blev indsat, så vi havde et overblik over de enkelte faser. Vi benyttede os af zenhub til dette, hvilket har haft en positiv indflydelse på styringen af projektet, da vi hele tiden har haft den opdaterede tidsplan. Da anvendelsen af Scrum var et krav fra projektets side, har denne metode også haft indflydelse på hvordan vi har planlagt projektet, som også blev beskrevet under metoder.

\subsubsection{Hvad ville vi gøre anderledes?}
Hvis vi skulle starte forfra med projektet, ville vi 

Mere tid på implementering i starten af implementeringsprocessen fra størstedelen af gruppen

Fulgt vores burndownchart for at komme mere clean i mål


\subsection{Individuelle evalueringer}
Nedenstående afsnit handler om, hvordan det enkelte gruppemedlem evaluere processen, udover det der er beskrevet i gruppeevalueringen. Her kan der blandt andet læses om, hvordan medlemmerne selv evaluerer deres egen indsat og deres læringsmæssige udbytte.  

\subsubsection{Jakob Rasmussen}
Hvordan har min egen indsat været?
Hvad har jeg lært?
Hvad har jeg ikke lært, som jeg måske burde have lært?
Hvordan synes jeg at gruppestørrelsen har påvirket læringen fra semesterprojektet?


\subsubsection{Kenneth Christiansen}
Hvordan har min egen indsat været?
Hvad har jeg lært?
Hvad har jeg ikke lært, som jeg måske burde have lært?
Hvordan synes jeg at gruppestørrelsen har påvirket læringen fra semesterprojektet?

\subsubsection{Kevin Petersen}
Hvordan har min egen indsat været?
Jeg synes jeg godt kunne have vœret mere igang til at starte med, til slut var jeg okay med, synes det ar œrveligt det allerade er slut.
Hvad har jeg lært?
Jeg har lœrt at code er ikke altid bare at skrive, det er at finde ud af hvordan man løser det bedst.
Hvad har jeg ikke lært, som jeg måske burde have lært?
Jeg burde have lœrt hvordan api'et authenticator personer og giver det tilbage som de må vide af data.
Hvordan synes jeg at gruppestørrelsen har påvirket læringen fra semesterprojektet?
Jeg synes gruppe størrelsen godt kunne have gjort det lidt uoverskueligt, hvad de andre arbejder på og hvad deres kode gør.
Det kommer lidt til at lignde en blackbox.
Jeg synes dog jeg har vœret godt til at lœse hvad de har lavet og forstå det. Mangler dog at lœse lidt mere på api'et.
Der var mange ting jeg ikke vidste hvad var i python jeg var nød til at finde ud af først, men det gør jeg kan mentor de andre senere for at forstå koden.

\subsubsection{Kristian Jakobsen}
Hvordan har min egen indsat været?
Hvad har jeg lært?
Hvad har jeg ikke lært, som jeg måske burde have lært?
Hvordan synes jeg at gruppestørrelsen har påvirket læringen fra semesterprojektet?

\subsubsection{Mathias Rasmussen}
Som helhed vurderer jeg projektet til at være forløbet nogenlunde smertefrit. Da jeg ikke har meget kodeerfaring, og stadig kæmper med basale færdigheder, har det resultereret i meget ekstra tid spenderet på at lære og forstå design mønsteret, MVVM, som gruppen har valgt at benytte. Når dette er sagt, har jeg lært en masse ved at være skubbet ud i at skulle kode mere kompliceret og virkelighedsnær kode. Jeg har derudover lært, at det at arbejde i en gruppe med gruppemedlemmer, der har mere programmeringserfaring end jeg selv, kan være med til at forårsage tvivl omkring kvaliteten af egen kode. Set i bakspejlet burde jeg have været bedre til at spørge om hjælpe, i stedet for at bruge unødvendig lang tid på at finde frem til en mulig tilgang. 
Da jeg ikke har været involeret i Rest API'et og EPG Polleren, ville jeg gerne have været mere inde over disse. Dette ville have givet et bedre overblik over det fulde system, og ikke blot desktop-klienten.\\
Jeg synes generelt arbejdsprocessen har været fin, og de andre gruppemedlemmer har været gode til at hjælpe til, når der blev spurgt til råds. Jeg synes dog at gruppens størrelse har gjort det en smule besværligt at holde overblikket, specielt da enkelte personer alene har lavet subsystemerne Rest API og EPG Poller. 


\subsubsection{Simon Jørgensen}
Hvordan har min egen indsat været?
Hvad har jeg lært?
Hvad har jeg ikke lært, som jeg måske burde have lært?
Hvordan synes jeg at gruppestørrelsen har påvirket læringen fra semesterprojektet?
