\section{Diskussion}
Denne sektion skal indeholde:

\begin{itemize}
    \item Hvad er der opnået og hvad er der ikke opnået i projektet i forhold til det forventede som beskrevet i indledningen.
    \item Hvad er styrkerne og svaghederne ved jeres resultater.
    \item Kunne I have opnået bedre resultater.
\end{itemize}{}
\\
Ved starten af projektet, har gruppen haft en forventning om at få konstrueret en prototype til et fuldt funktionelt krediteringssystem. Denne prototype indebærer desktop-klient, Rest API og EPG Poller og skal kunne indeholde offentligt tilgængelige krediteringer. Man skal som producer af en tv-produktion kunne oprette krediteringer. Derudover skal alle kunne søge efter krediteringer. 
Generelt set er disse indlendende forventninger opnået. Gruppen har konstrueret en desktop-klient, der gør det muligt at oprette, søge efter og læse krediteringer for TV-produktioner listet på TVTID.dk. Denne data er hentet via en EPG Poller, og gemt i en database med Rest API som formidlingsstation.
\myworries{Figur (x)} viser MoSCoW-modellen, som er en opdeling og prioritering af de forskellige brugsmønstre systemet, optimalt set, skulle indeholde. Da prototypen ikke er fuldendt, er ikke alle disse brugsmønstre opnået.
Gruppen har i desktop-klienten ikke fået implementeret funktionalitet der bl.a. gør det muligt for brugeren at logge af, oprette brugere, se personinformationer og godkende/afvise nye krediteringer. Derimod er disse blevet implementeret i API'et, med undtagelse af 'log af' funktionalitet. Derudover er der lavere-prioriterede brugsmønstre der ikke er blevet implementeret, herunder bl.a. muligheden for at se personprofiler og oprette kanal- og systemadministrator. \\



\\
- Hvad forventede vi at opnå? \\ Vi forventede at konstruere en prototype til et fuldt funktionelt krediteringssystem, med Rest API, EPG Poller og desktop-klient. Denne prototype skulle kunne indeholde krediteringer så de var offentligt tilgængelige, oprette krediteringer og søge efter krediteringer. (SE MoSCoW)   - Hvad er opnået? \\ Vi opnåede det forventede i indledningen (ikke alt fra MoSCoW) \\




- Hvad er ikke opnået? \\
Det meste blev implementeret i API'et, men ikke i klienten.
Brugerroller, se personinformation, godkend/afvis nye krediteringer, fjern kreditering, ændre sprog

Ikke i desk: \\
Must:   B03, B14, B15, B16
Should: B01, B02, B05, B11, B17, B18, B19
Could:  B20

Ikke i EPG poller:
Docker på timer (kør 1 gang i døgnet)

Ikke i api: \\ TODO(I still plan on getting this done (((-:))) one day 2030.
#80, #81, #83, #84 - se zenhub

- Styrker ved resultaterne
Nem at udvide - MVVM - Det meste er separeret, så der er ikke mange dependencies 
Ikke låst på én type klient - læs nedenstående
API - Hvis TV2 hellere vil have en anden klient fx en hjemmeside så kan de genanvende API'et.
EPG Poller - Henter allerede eksisterende data, så det ikke skal "genskabes".

- Svagheder ved resultaterne?
At produktioner ikke er opdelt i sæsoner og episoder? - Alle produktioner er i én lang liste

Det tager tid at sætte sig ind i strukturen - MVVM

Til API'et er der implementere mange unittest men coverage rapporten fra sonarcloud reflekterer ikke dette (0\% coverage)
JavaFX - Svært at teste, samt ikke ønsket fra TV2's side (en hjemmeside vil være en del af deres faktiske slutprodukt). #DesktopKlienterI2020


- Bedre resultater?
Slut produktet vil være bedre hvis vi i stedet havde lavet et website i stedet for en desktop klient. Selvom dette er et pilot projekt, vil TV2's løsning med garanti bruge et website som klient, da clearingsprocessen og udrulningen af desktop klienter til deres medarbejdere vil være en tung proces.


