\section{Diskussion}

\\
Ved starten af projektet, har gruppen haft en forventning om at få konstrueret en prototype til et fuldt funktionelt krediteringssystem. Denne prototype indebærer desktop-klient, Rest API og EPG Poller, og skal kunne indeholde offentligt tilgængelige krediteringer. Man skal som producer af en tv-produktion kunne oprette krediteringer. Derudover skal alle kunne søge efter krediteringer. \\
Generelt set er disse indledende forventninger opnået. Gruppen har konstrueret en desktop-klient, der gør det muligt at oprette, søge efter og læse krediteringer for TV-produktioner listet på TVTID.dk. Denne data er hentet via en EPG Poller, og gemt i en database med Rest API som formidlingsstation. Projekt beskrivelsen lægger op til at systemet, optimalt set, skal bygges som en webapplication, men grundet at tidligere undervisning har været fokuseret på JavaFX og Scene Builder, har gruppen valgt at konstruere en desktop-klient.\\
Inceptionsdokumentets bilag \ref{inceptionsdokument} viser MoSCoW-modellen, som er en opdeling og prioritering af de forskellige brugs\-møn\-stre systemet, optimalt set, skulle indeholde. Da prototypen ikke er fuldendt, er ikke alle disse brugsmønstre opnået.
Gruppen har i desktop-klienten ikke fået implementeret funktionalitet der bl.a. gør det muligt for brugeren at logge af, oprette brugere, se personinformationer og godkende/afvise nye krediteringer. Derimod er disse blevet implementeret i API'et, med undtagelse af 'log af' funktionalitet. De forskellige brugerroller (System- og kanaladministrator, royalty bruger og producer) er ligeledes kun implementeret i API'et. Disse brugerroller vil skulle validere hvorvidt visse knapper og funktionaliteter er tilgængelige i klienten, ud fra hvilke rettigheder den pågældende brugerrolle har.
Derudover er der lavere-prioriterede brugsmønstre der ikke er blevet implementeret, herunder bl.a. muligheden for at se personprofiler og oprette kanal- og systemadministrator.\\ 
EPG Polleren skal gerne køre én i døgnet, for at hente nye produktioner og de tilhørende krediteringer. Dette er dog ikke implementeret, og skal derimod køres manuelt. \myworries{Dertil er planen, at nye krediteringer skal oprettes direkte i klienten, hvilket gør at EPG Polleren kun skal bruges til at hente nye kanaler og produktioner. Ikke nye krediteringer}.\\

Systemets lagdeling er modelleret ud fra et design mønstret, MVVM. Denne lagdeling gør at det meste af koden er separeret i forskellige pakker, og giver derved en lav afhængighed mellem disse. Dette gør dog også at det kan være besværligt at udvide programmet med ny funktionalitet, da der skal implementeres mange forskellige metoder og klasser. Samtidig kan MVVM strukturen være tidskrævende at sætte sig ind i. Systemets opdeling gør at man nemt kan udvide med en ny slags/form for klient, fx en webside eller en web app. Dette skyldes at pakkerne kommunikerer igennem interfaces og metodekaldene foregår til API'et, hvilket gør det muligt at udskifte klienten. \\
Som produktionerne er inddelt nu, er hver episode i en serie tildelt sin egen produktion. Optimalt set vil hver produktion indeholde tilhørende sæsoner og episoder, som hver indeholder dertilhørende krediteringer. 


