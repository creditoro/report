\section{Resume}
Denne sektion skal indeholde:

\begin{itemize}
    \item Hovedresultater og konklusioner  – hvad kom der ud af arbejdet
\end{itemize}{}

I dette projekt bliver der arbejdet ud fra TV2's problemstilling omhandlende muligheden for at flytte krediteringer fra TV til en anden platform. \myworries{Denne plads kan så udfyldes med andet som fx. reklamer og promoveringer for egne programmer.} Gruppen har valgt at bygge et fuldt funktionelt program, inklusive Rest API og EPG poller. Dertil er gruppen fundet frem til en problemformulering der omhandler udviklingen af et krediteringssystem, hvordan krediteringerne skal gøres tilgængelige samt håndteres. Derudover undersøges det hvordan systemet kan indeholde krediteringerne.
Projektgruppen har valgt at afgrænse projektet ved at lave en prototype til et færdigt program. Ideelt ville systemet også have en webside, dog er dette vurderet til værende udenfor projektets scope.

Overordnet benyttes UP (Unified Process) i hele projektet. UP opdeler hele projektet i 4 faser: \textit{Inceptionsfasen} hvor vigtige krav og kristiske risici identificeres. \textit{Elaborationfasen} hvor en iterativ udvikling af krav, design, analyse og test konstrueres ud fra overordnede kravspecifikationer. \textit{Konstruktionfasen} hvor systemet konstrueres - den faktiske kodeudformning. \textit{Overgangsfasen} hvor det undersøges om systemet er færdigt og ikke har mangler.

I starten af projektets analysefase er et brugsmønsterdiagram, herunder aktørliste og brugs\-møn\-ster\-li\-ste, udformet. Dette er med til at give et billede af hvordan systemet skal bygges op, samt hvilke aktører der skal kunne interagere med hvilke brugsmønstre. Til prioritering af brugsmønstrene er der foretaget en MoSCoW-analyse.
Til identificering af ikke-funktionelle krav er der brugt metoden FURPS. Her er der sat krav op der omhandler Functionality, Usability, Reliability, Performance og Supportability. For yderligere specificering af ikke funktionelle krav er FURPS+ benyttet - FURPS med ekstra specificerende kategorier med til at opfylde kundens behov.

Til projektstyring benyttes Scrum til at fordele og få overblik over de forskellige opgaver, der opdeles i mindre issues. De forskellige arbejdsopgaver inddeles i sprints som forløber sig over en given periode - i dette projekts tilfælde to uger.

\myworries{Hovedresultater og konklusioner}


