\section{Konklusion}

Det lykkedes gruppen at udvikle en prototype til et krediteringssystem, der giver mulighed for at erstatte rulletekster, efter endt TV-program. Med dette system kan man søge efter og finde krediteringer for programmer på forskellige kanaler, samt oprette nye krediteringer. Disse krediteringer skal være offentligt tilgængelige, og kunne redigeres og tilføjes af producere. Systemets opdeling af pakker følger design mønstret, MVVM, og gør det muligt at udskifte klienten til fx en website/webapplikation, da klienten kalder til et Rest API og gemmer i en database. Denne opdeling gør også at klasserne har en lav afhængighed. Allerede eksisterende data kan hentes via en EPG Poller fra TVTID.dk.
De indledende forventninger er at gruppen har kunnet lave en fuldt funktionel prototype, hvilket er blevet opnået. Enkelte funktionaliteter mangler dog i desktop-klienten. \myworries{Denne prototype er udviklet ved at følge Unified Process modellen og de tilhørende faser. Udformningen af kravspecifikationen og analysen viser hvilke krav og brugsmønstre systemet skal bygges op om, for at opfylde TV2's projektbeskrivelse.}\\
\myworries{INDSÆT NOGET OM DESIGN DELEN} \\
\myworries{INDSÆT NOGET OM TEST DELEN} \\
Gruppen har, ud fra de udarbejdede resultater i krav, analyse og design, udviklet prototypen til et krediteringssystem.\\

\\

\myworries{- Hvordan skal folk refereres = ????????} \\ %<--------------------------------------------