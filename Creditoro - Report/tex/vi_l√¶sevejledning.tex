\section{Læsevejledning}

Rapporten er disponeret sådan, at \textit{afsnit 1 - 3} beskriver projektets udgangspunkt. Herunder findes det essentielle problem rapporten er udformet ud fra, de benyttede metoder og værktøjer igennem hele projektet, planlægningen og strukturering samt det faglige vidensgrundlag.

\textit{Afsnit 5-6} indeholder kravspecifikationer herunder udarbejdelsen af de overordnede krav. Derudover kan de detaljerede krav findes her.

Analysen af projektet findes i \textit{afsnit 7 - Analyse}, hvor der bliver udformet en brugsmønsteranalyse og brugsmønsterrealisering samt et resumé af konstrueringen af EAST-ADL modeller ud fra funktionelle og ikke funktionelle krav.

I \textit{Afsnit 8-9} kan der læses om projekts designfase. Herunder findes systemets design samt systemets subsystemer. Det er her også muligt at se designafsnittet for databasen.

I \textit{afsnit 10 - Implementering} kan der læses om konverteringen fra design til kode, samt vigtige implementeringsbeslutninger. Dette afsnit dækker hele systemet inklusiv subsystemer.

\textit{Afsnit 11 - Test} indeholder tests af en række udvalgte metoder og klasser fra systemet i henholdsvis 1. og 2. iteration.

Rapporten rundes af med \textit{afsnit 12 - 15}, hvor diskussion, konklusion, perspektivering samt procesevaluering er at finde. I disse afsnit vil der blive fokuseret på, hvad der opnået i projektet, hvad der kan gøres bedre, opsummering af resultater, samt en evaluering over hele projektets proces.

Vejleder- og samarbejdsaftale vil være at finde i \textit{bilag \ref{vejlederaftale}} og \textit{\ref{samarbejdsaftale}}.

Kun udvalgte modeller, tabeller og diagrammer er indsat i rapporten, resten vil være at finde i \textit{bilag}.

Ønskes det at læses om systemets subsystemer - herunder Rest API og EPG Poller - vil dette være at finde i \textit{bilag}. Her findes analyse, design, implementering og tests af de to subsystemer.
