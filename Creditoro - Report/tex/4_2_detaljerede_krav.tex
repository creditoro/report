\section{Detaljerede Krav}

Detaljerede krav er en uddybelse af foregående afsnit \ref{overordnede_krav} "Overordnede Krav". I dette afsnit vil der udarbejdes detaljerede brugsmønsterbeskrivelser udfra de allerede udarbejdede overordnede brugsmønsterbeskrivelser. Der vil I dette afsnit blive fokuseret på det mest essentielle brugsmønster for systemet.

Afsnittet vil derudover indeholde en detaljeret beskrivelse af de supplerende ikke-funktionelle krav. Til dette bliver FURPS+ modellen benyttet.


% ---------------------------- Detaljerede brugsmønstre ----------------------------
\subsection{Detaljerede Brugsmønstre} {\label{section: detailed_usemodels}}
I tabel \ref{tab:create_credits} ses en detaljeret beskrivelse af det essentielle brugsmønster "Opret Kreditering".
Beskrivelsen er udarbejdet ud fra aktørlisterne og brugsmønster beskrivelserne i inceptionsdokumentet (se bilag \ref{inceptionsdokument} afsnit 3). Detaljerede brugsmønster beskrivelserne bruges til at danne et overblik over hver enkel brugsmønsters cyklus. Her nævnes brugsmønstrets primære og sekundære aktører, de præ- og postkonditioner der er knyttet til brugsmønsteret - hvad skal være opfyldt, for at brugsmønsteret sættes i gang, og hvad sker der når brugsmønsteret er gennemført. Herudover beskrives hoved- og alternative hændelsesforløb. Hovedhændelsesforløbet beskriver forløbet, hvis der ingen problemer opstår. Alternativ hændelsesforløb beskriver eventuelle alternative hændelser i specifikke trin/steps i hovedhændelsesforløbet. \\

Brugsmønstrene nedenfor er blevet valgt på baggrund af MosCow analysen i inceptionsdokumentet (se bilag \ref{inceptionsdokument} side 15) baseret på systemets brugsmønstre. Det mest essentielle brugsmønster er valgt ud fra brugsmønstrene i kategorien "must have". Det er vurderet at brugsmønstret "Opret Kreditering" er det mest essentielle bnrugsmønster, da det er denne funktion systemet er bygget op omkring.\\

% % ----------------------- Opret krediteringer -----------------------------------
 \subsubsection{Opret Krediteringer}
\begin{longtable}[h] {|p{16cm}|}
 \hline
       \textbf{Brugsmønster:} Opret kreditering \\
    \hline
	\textbf{ID:} UC07 \\ \hline
	\textbf{Primære aktører:} Systemadministrator, kanaladministrator, Producer \\ \hline
	\textbf{Sekundære aktører:} \\ \hline
 	\textbf{Kort beskrivelse:} Produceren opretter en kreditering. Heri angives alle der har bidraget til produceringen af TV-programmet, filmen el. lign. \\ \hline
	\textbf{Prækonditioner (Pre conditions):} \\
 Aktøren skal være \hyperref[table:login]{logget ind på} systemet \\ \hline
 \textbf{Hovedhændelsesforløb (main flow):} \\
	1. Brugsmønstret starter når en administrator eller producer vil oprette en kreditering \\
	2. Aktøren trykker på knappen ‘Opret Kreditering’ \\
	3. Systemet checker aktørens rolle \\
	4. Aktøren er forbundet til en kanal, og angiver programmets titel \\
	5. Systemet checker om der allerede findes et program med den angivne titel \\ 
	6. Aktøren krediterer alle der har medvirket i produktionen af programmet \\ 
	7. Aktøren sender den færdige kreditering videre til godkendelse \\ \hline
\textbf{Postkonditioner (post conditions):} \\ 
	En kreditering er blevet sendt videre til godkendelse \\ \hline
	\textbf{Alternative hændelsesforløb (alternative flow):} \\
tep *: Aktøren kan til enhver tid afbryde oprettelsen af krediteringen \\
Step 4: Hvis aktøren er systemadministrator, er vedkommende ikke forbundet til en kanal, og kan skifte hvilken kanal krediteringen skal oprettes ved. \\

Step 5: Hvis programmets titel allerede eksisterer, gøres aktøren opmærksom på dette. \\

Step 8: Hvis krediteringen afvises, laves de fornødne ændringer, og den nye kreditering sendes videre til godkendelse. \\
\hline
\caption{Brugsmønster: Opret kreditering}
\label{tab:create_credits}
\end{longtable}


% ---------------------------- Detaljeret beskrivelse af supplerende krav ------------------------------------
\subsection{Detaljeret beskrivelse af supplerende krav}

 I tabel \ref{tab:furps+} ses de detaljerede beskrivelser af supplerende ikke-funktionelle krav i form af en FURPS+ model. modellen er med til at give et overblik over hvilke supplerende ikke-funktionelle krav systemet skal overholde, for at være tilstrækkeligt i forhold til TV2 og gruppens forventninger.
FURPS+ er en udviddet udgave af FURPS modellen. Udover kravene i FURPS modellen i afsnit \ref{supplerende_krav}, er der i tilføjet 4 nye kategorier - Design constraints, Implementation requirements, Interface requirements og Physical requirements. Derudover har hver af de ikke-funktionelle krav fået tildelt et nummer for lettere reference. \\
Af de tilføjede kategorier er nye krav udformet. Systemet skal benytte en relationel databse. Databasen skal skrives i programmeringssproget SQL og laves i PostgreSQL. Desktop clienten bygges i programmeringssproget Java og REST Api'et bygges i Python.\\
Systemet vil blive unit testet, for at sikre en høj kvalitet af implementeret kode. Systemet vil blive forbindet til det centraliserede fejlognings system \texttt{Sentry}.\\
Systemet skal kunne trække EPG data fra TVTid.dk via en EPG Poller og sende dataen til REST Api'et.

\begin{center}
    \begin{longtable}[h]{|p{4cm}|p{1cm}|p{11cm}|}
        \hline
        \textbf{FURPS+}                     & \textbf{\#}   & \textbf{Krav} \\ 
        \hline
        \multirow{3}{*}{Functionality}      & \multirow{2}{*}{S01}           & Skal kunne kreditere produktionsroller som er angivet af DRs krediteringsregler.  \\ \cline{2-3}
                                            & S02           & Skal overholde GDPR. \\ \hline
        \multirow{2}{*}{Usability}          & S03           & Systemet skal kunne understøtte flere sprog. \\ \cline{2-3}
                                            & S04           & Systemet skal have en responsiv brugergrænseflade (UI). \\ \hline
        \multirow{3}{*}{Reliability}        & \multirow{3}{*}{S05}           & Hvis serveren til systemet genstarter, startes del-systemerne igen automatisk. Der vil ikke være behov for at ligge systemet ned regelmæssigt for at kunne foretage backup. \\ \hline
        \multirow{4}{*}{Performance}        & \multirow{3}{*}{S06}           & Databasen skal kunne håndtere 10000 nye brugere - samt 15000 krediteringerer årligt i 25 år, uden at ofte brugte kald til REST Api'et bliver sløvt (reponsetid på mere end 300 ms). \\ \cline{2-3}
                                            & S07           & Systemet skal kunne håndtere 5000 brugere indenfor et minut.\\ \hline
        \multirow{4}{*}{Supportability}     & S08           & Systemet er installerbart vha. Docker via Docker-compose.\\ \cline{2-3}
                                            & \multirow{3}{*}{S09}           & Det vil være muligt at konfigurere system indstillinger via en \texttt{.env} (miljø) fil. En opsætningsguide vil være at finde sammen med kildekoden. \\ \hline
        \multirow{3}{*}{Design constraints} & \multirow{3}{*}{S10}           & Systemet skal benytte en relationel database. Databasen skal laves i PostgreSQL og implementeres i programmeringssproget SQL.\\ \hline
        \multirow{6}{}{Implementation requirements}   & \multirow{2}{*}{S11}   & Systemet vil indeholde unit tests, og komme med en rapport over hvor stor en procentdel der er dækket af dette.\\ \cline{2-3}
                                            & \multirow{2}{*}{S12}           & Systemet vil blive forbundet til det centraliserede fejllognings system \texttt{Sentry}.\\ \cline{2-3}   
                                            & \multirow{2}{*}{S13}           & Desktop klienten implementeres i programeringssproget Java. REST API implementeres i Python.\\ \hline
        \multirow{2}{*}{Interface requirements}              & \multirow{2}{*}{S14}          & Systemet skal trække EPG data fra TVTid.dk via en EPG Poller. \\ \hline
        Physical requirements               & S15           & Ingen.\\ 
        \hline
    \caption{FURPS+}
    \label{tab:furps+}
    \end{longtable}
\end{center}