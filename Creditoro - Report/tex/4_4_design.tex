\section{Design}
Denne sektion skal indeholde:

\begin{itemize}
    \item Overvejelser, beslutninger og resultater vedr. Softwarearkitektur, Subsystemdesign, design af persistens. 
\end{itemize}{}

\subsection{Persistens design}
Da persistens skulle designes, var der en stor overvejelse om hver klient skulle have deres egen database, eller om der skulle stræbes efter at få et mere virkelighedstro system (et system der vil afspejle det TV2 lagde op til), der indeholder en centraliseret database. Vi valgte at stræbe efter det sidstnævnte, og i forbindelse med dette blev der foretaget et valgt at kommunikationen mellem klient og database, skulle foregå gennem et REST API så klienterne ikke har direkte adgang til vores database, på denne måde kan vi også sikre os at brugere af systemet, kun kan foretage de handlinger som deres respektive rolle giver adgang til, da validationen ligger på server-siden (så brugeren ikke bare kan sniffe brugernavn - og kode til databasen og få adgang af andre veje).


\subsection{Subsystemdesign}


\subsubsection{API}


\subsubsection{Desktop Klient}
    

\subsubsection{resultater}


\subsection{Softwarearkitektur}        
\subsubsection{Overvejelser}
\myworries{hvilke overvejelser havde vi i forhold til strukturen af hele systemet og hvordan kom vi frem til diagramet} \\
Gruppen startede med at kigge projekte caset igennem. \cite{TV2-case} \hyperlink{https://docs.google.com/document/d/1p6lQjWV76TX9uTLst2OAdmV07XfMRn5fObelzBpd2EI/edit}{Projekt case} var det første der blev omtalt. Der blev snakket meget om vi kunne nå at lave et Rest API.


\subsubsection{Beslutninger}


\subsubsection{resultater}

\begin{figure}[h]
    \centering
    \includegraphics[width=\textwidth]{figures/design/domaindiagram-Software Architecture Diagram.png}
    \caption{Domaindiagram Software Architecture Diagram}
    \label{fig:domaindiagram-software-diagram}
\end{figure}{}