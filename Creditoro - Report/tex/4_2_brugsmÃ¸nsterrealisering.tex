\section{Brugsmønsterrealisering}

% -----------------------------------------------------------------------------------------------------------------
\subsection{Systemsekvensdiagram}
Et systemsekvensdiagram er et sekvensdiagram der viser systemhændelserne for ét scenarie af et
brugsmønster. Diagrammet viser hvordan aktørerne interagerer med systemet for at opfylde
brugsmønstret. Diagrammet viser systemet som en ‘black box’, hvilket betyder at man ikke kan se
hvad der sker inde i systemet, man kun hvad der sker udenfor systemet. På diagrammet ses det,
hvordan aktørerne generere systembegivenheder og hvad systemets output er. Ydermere viser
diagrammet den ‘tidslinje’ begivenhederne sker i. \\

\noindent
\myworries{Indsæt tekst om hvilke systemsekvensdiagrammer vi har valgt at medbringe her i teksten, hvorfor og at de andre kan findes i bilag :-) - Sammenkobel den her tekst med den tekst der er i brugsmønsterrealisering - kechr}

\begin{figure}[h]
\centering
\includegraphics[scale=0.43]{figures/systemsekvensdiagrammer/læsKreditering.PNG}
\caption{Systemsekvensdiagram for "Læs kreditering"}
\label{fig:read_credit}
\end{figure}


% -----------------------------------------------------------------------------------------------------------------
\subsection{Kontrakter for systemfunktioner}
En systemoperationskontrakt beskriver en operations ansvar, altså hvad en operation har forpligtet
sig til. Kontrakten lægger vægt på hvad en operation ændre på, og ikke på hvordan det
ændre sig. En kontrakt kan derfor anses som værende en formel beskrivelse af en operation.\\
Systemoperationskontrakten indeholder navnet på operationen, krydsreferencer til til de relevante
brugsmønstre, beskrivelse af ansvaret og det output operationen genererer, og pre- og
postkonditioner for operationen. \\
Pre- og postkondinitionerne er stilbilleder af systemet på det givne tidspunkt operationen bliver
kaldt. De beskriver altså systemets tilstand før og efter at operationen har kørt. Postkonditioner skal
altid noteres i datid, som f.eks. “Købet blev foretaget”, da det er en ting der er sket, og ikke sker.\\

\noindent
Det er valgt at lave operationskontrakter for "Læs kreditering", "Opret kreditering", "Opret producer", "Se personinformation" og "Godkend ny kreditering", da disse operationer viser essensen af den logik der skal implementeres. \myworries{Skriv dette om - Sammenkoble med den tekst der står under systemsekvensdiagrammer -- Eventuelt skriv det under 10. brugsmønsterrealisering, at vi har valgt at arbejde med de brugsmøsnter/operationer/we vi har - begrund uddybende - kechr}


%------------------------ Læs krediteringer -------------------------------
\subsubsection{Læs Krediteringer}
\begin{table}[h]
    \centering
    \begin{tabular}{| p{4cm} | p{12cm} |}
    \hline
    \multicolumn{2}{|c|}{\textbf{Læs Kreditering}}\\
    \hline
    \textbf{System operation}       & \textbf{læsKreditering} \\ \hline
    \textbf{Krydshenvisning}        & Use case: Læs kreditering \\ \hline
    \textbf{Ansvar}                 & At vise eksisterende krediteringer, hvis følgende betingelser er sande \\ 
                                    & \\
                                    & \quad 1. Den søgte kreditering findes i systemet\\
                                    & \\
                                    & Hvis ikke overstående er sande, skal det sendes en besked til brugeren at den søgte kreditering ikke findes i systemet\\\hline
    \textbf{Output}                 & 1. Krediteringen vises\\ 
                                    & Alternativt: Besked sendt ud til brugeren\\ \hline
    \textbf{Prækonditioner}         & Ingen \\ \hline
    \textbf{Postkonditioner}        & Du vil have læst kreditering \\ \hline
    \end{tabular}
    \caption{Systemfunktionskontrakt 'læsKreditering'}
    \label{tab:kontrakter_læs_kreditering}
\end{table}


%------------------------ Opret kreditering -------------------------------
\subsubsection{Opret Krediteringer}
\begin{table}[h]
    \centering
    \begin{tabular}{| p{4cm} | p{12cm} |}
    \hline
    \multicolumn{2}{|c|}{\textbf{Opret kreditering}}\\
    \hline
    \textbf{System operation}       & \textbf{opretKreditering} \\ \hline
    \textbf{Krydshenvisning}        & Use case: Opret Kreditering \\ \hline
    \textbf{Ansvar}                 & At oprette krediteringer og sende den videre til godkendelse hvis prækonditionen er opfyldt, og                                       følgende betingelser er sande: \\
                                    & \\
                                    & \quad 1. Minimumskravene er opfyldt\\
                                    & \quad 2. Alle oplysninger er indtastet korrekt \\
                                    & \\
                                    & Hvis ikke ovenstående er sande, oprettes krediteringen ikke, og brugeren bliver informeret herom \\ \hline
    \textbf{Output}                 & \quad 1. Produceren får besked om krediteringen er oprettet \\ 
                                    & \quad 2. System- og/eller kanaladministrator får besked om en nyoprettet kreditering \\\hline
    \textbf{Prækonditioner}         & Logget ind som kanal- eller systemadministrator \\ \hline
    \textbf{Postkonditioner}        & En ny kreditering er oprettet i systemet \\ \hline
    \end{tabular}
    \caption{Systemfunktionskontrakt 'Opret kreditering'}
    \label{tab:kontrakter_opret_kreditering}
\end{table}


%------------------------ Opret producer -------------------------------
\subsubsection{Opret Producer}
\begin{table}[H]
    \centering
    \begin{tabular}{| p{4cm} | p{12cm} |}
    \hline
    \multicolumn{2}{|c|}{\textbf{Opret producer}}\\
    \hline
    \textbf{System operation}       & \textbf{opretProducer} \\ \hline
    \textbf{Krydshenvisning}        & Use case: Opret producer \\ \hline\textbf{}
    \textbf{Ansvar}                 & At oprette en producer, hvis prækonditionen er opfyldt og følgende betingelser er sande: \\ 
                                    & \\ 
                                    & \quad 1. Produceren eksisterer ikke i systemet i forvejen \\
                                    & \quad 2. Alle oplysninger er indtastet korrekt \\
                                    & \\
                                    & Derefter give brugeren besked om det lykkedes at oprette en producer eller ej \\\hline
    \textbf{Output}                 & Besked om der er blevet oprettet en producer eller ej \\ \hline
    \textbf{Prækonditioner}         & Vedkommende der prøver at oprette en producer skal være logget ind som 'Kanal- eller                                                  Systemadministrator' \\ \hline
    \textbf{Postkonditioner}        & En ny producer er oprettet i systemet \\ \hline
    \end{tabular}
    \caption{Systemfunktionskontrakt 'Opret producer'}
    \label{tab:kontrakter_opret_producer}
\end{table}


%------------------------ Se Personinformationer -------------------------
\subsubsection{Se Personinformationer}
\begin{table}[H]
    \centering
    \begin{tabular}{| p{4cm} | p{12cm} |}
    \hline
    \multicolumn{2}{|c|}{\textbf{Se Personinformationer}}\\
    \hline
    \textbf{System operation}       & \textbf{sePersonInfo} \\ \hline
    \textbf{Krydshenvisning}        & Use case: Se personinformation \\ \hline
    \textbf{Ansvar}                 & At vise respektive persondata om den søgte person, hvis prækonditionen og betingelsen er opfyldt:\\
                                    & \\
                                    & \quad 1. Personen er logget ind som enten producer, kanal- eller systemadministrator\\
                                    & \\
                                    & Hvis ikke overstående er opfyldt, vises ikke personfølsomme informationer, som navn, roller i film/serier som personen har medvirket i\\ \hline
    \textbf{Output}                 & Viser en information om en person \\ \hline
    \textbf{Prækonditioner}         & Person er fundet i systemet \\ \hline
    \textbf{Postkonditioner}        & Alt efter ens rolle i systemet vil forskelligt data blive vist. Begrænset hvis man er besøgende, alt hvis man er producer, royalty bruger, kanal- og systemadministrator \\ \hline
    \end{tabular}
    \caption{Systemfunktionskontrakt 'se personinformationer'}
    \label{tab:kontrakter_se_personinformationer}
\end{table}

%------------------------ Godkend nye krediteringer -------------------------
\subsubsection{Godkend nye Krediteringer}
\begin{table}[H]
    \centering
    \begin{tabular}{| p{4cm} | p{12cm} |}
    \hline
    \multicolumn{2}{|c|}{\textbf{Godkend ny kreditering}}\\
    \hline
    \textbf{System operation}       & \textbf{godkendKreditering} \\ \hline
    \textbf{Krydshenvisning}        & Use case: Godkend Kredtiteringer \\ \hline
    \textbf{Ansvar}                 & At godkende eller afvise nye krediteringer hvis prædonditionerne er sande og betingelsen opfyldt:  \\ 
                                    & \\
                                    & \quad 1. En kreditering er oprettet af en producer \\
                                    & \\
                                    & Hvis ikke ovenstående er opfyldt, vil der ikke blive vist nye krediteringer til godkendelse. \\ \hline
    \textbf{Output}                 & Besked om krediteringen er godkendt \\ \hline
    \textbf{Prækonditioner}         & 1. Er logget ind som system- eller kanaladministrator \\ \hline
    \textbf{Postkonditioner}        & Krediteringen er enten godkendt eller afvist og den ansvarlige producer er bliver informeret herom \\ \hline
    \end{tabular}
    \caption{Systemfunktionskontrakt 'Godkend nye krediteringer'}
    \label{tab:kontrakter_Godkend_nye_krediteringer}
\end{table}



% -----------------------------------------------------------------------------------------------------------------
\subsection{Operationssekvensdiagram}
Operationssekvensdiagrammet viser systemet som en ”white box”, hvor man kan se hvad der sker
inde i systemet. Sekvensdiagrammet bruges til at identificere systemfunktioner, da de begivenheder der vises i
diagrammet er de funktioner systemet skal indeholde.

\begin{figure}[h]
\centering
\includegraphics[scale=1]{figures/Operationssekvensdiagrammer/læsKreditering.pdf}
\caption{Operationssekvensdiagram: Læs kreditering}
\label{fig:op_read_credit}
\end{figure}

% -----------------------------------------------------------------------------------------------------------------
\subsection{Revurderet analysemodel}


\myworries{Der blev ikke fundet nogel nye klasser eller metoder under brugsmønsterrealiseringen :)}