\section{Test - REST api \& EPG-Poller}\label{api_epg_test}

{\large\textbf{1. Iteration af api testning}}

\myparagraph{Test af Users klassen}
UsersTest klassen importerer metoder fra klassen BestTestCase. Metoden test\_all er den der foretager testen på users ved at bruge metoderne i BestTestCase.\\

\begin{code}[language=python, caption=UsersTest.py, firstnumber=1]
from http import HTTPStatus

from tests.base_test import BaseTestCase

class UsersTest(BaseTestCase):
    path = "users"

    def test_all(self):
        self.get_list()

        response = self.post(json=self.json())
        identifier = response.json.get("identifier")
        self.patch(identifier=identifier, json=self.patch_json())
        self.get(identifier=identifier
        self.put(identifier=identifier, json=self.json())
        self.delete(identifier=identifier)
\end{code}

Der startes med at kalde get\_list på self, som er en metode importeret fra BestTestCase og ser således ud:\\

\begin{code}[language=python, caption=BestTestCase.py, firstnumber=55]
def get_list(self):
    response = self._get(path=f"/{self.path}/")
    self.assertTrue(response.status_code == HTTPStatus.OK)
    return response
\end{code}

Herefter bliver der lavet et response objekt, som bliver instantieret ved at kalde post metoden i BestTestCase med en parameter der bliver dannet i json metoden fra UsersTest klassen. \\

\begin{code}[language=python, caption=UsersTest.py, firstnumber=12]
response = self.post(json=self.json())
\end{code}

\begin{code}[language=python, caption=UsersTest.py, firstnumber=19]
def json(self, phone: str = None, email: str = None, name: str = None, password: str = None):
    password = password or self.random_string()
    return {
        "phone": phone or "+45 12 12 12 12",
        "email": email or f"{self.random_string()}@creditoro.nymann.dev",
        "name": name or self.random_string(),
        "password": password,
        "repeated_password": password
    }
\end{code}

Json har muligheden for at tage parametre, men behøver det ikke. Så hvis man ikke ønsker en specifik json får man en forudindstillet json. Det json der bliver oprettet har et telefonnummer, email, navn, password og repeated\_password.
Efter json er sat op kan det blive videregivet til post metoden i BestTestCase:\\

\begin{code}[language=python, caption=BestTestCase.py, firstnumber=65]
def post(self, data: dict = None, json: dict = None):
    response = self._post(path=f"/{self.path}/", data=data, json=json)
    self.assertTrue(response.status_code == HTTPStatus.CREATED)
    return response
\end{code}

Post metoden starter med at oprette et repsons objekt, ved at kalde metoden \_post med det json der blev videregivet. \\

\begin{code}[language=python, caption=BestTestCase.py, firstnumber=40]
def _post(self, path: str, data: dict = None, json: dict = None):
    return self.client.post(path, headers=self.headers, json=json, data=data)
\end{code}

I \_post metoden der kaldt en post metode på klienten, der returnerer et respons objekt.\\
I post metoden bliver respons objektet sat til det returnerede responsobjekt. Herefter bliver der testet om at HTTPStatus er CREATED ved brug af assertTrue der tjekker om response objektets status\_code er det samme som OK. Eller fejler testen. Og til sidst returnerer response objektet til test\_all metoden.\\

Efter at have oprettet response objektet, bliver der oprettet et identifier objekt. Dette objekt bliver sat til response objektets identifier værdi. Identifier bliver herefter brugt til at kalde patch metoden, der tager en identifier og værdien som patch\_json metoden returnere som parametre. \\

\begin{code}[language=python, caption=UsersTest.py, firstnumber=14]
self.patch(identifier=identifier, json=self.patch_json())
\end{code}

\begin{code}[language=python, caption=UserTest.py, firstnumber=29]
def patch_json(self, name: str = None):
    return {
        "name": name or self.random_string()
    }
\end{code}

Patch\_json ændrer enten navnet til noget bestemt eller tilfældigt, og returnerer det reviderede json objekt. De to parametre, identifier og json, bliver så videre givet til patch metoden i base\_test.\\

\begin{code}[language=python, caption=BestTestCase.py, firstnumber=70]
def patch(self, identifier: str, data: dict = None, json: dict = None):
    response = self._patch(path=f"/{self.path}/{identifier}", data=data, json=json)
    self.assertTrue(response.status_code == HTTPStatus.OK)
    return response
\end{code}

I patch metoden bliver der oprettet et nyt response objekt ved at kalde \_patch metoden. \_patch metoden returnere, ligeseom \_post metoden, et response objekt. Response objektet bliver igen testest igennem assertTrue metoden, så hvis responsens status\_code er OK, består testen.\\

Efter patch metoden bliver kaldt, bilver get metoden kaldt. Get metoden tager identifier objektet som parameter. \\

\begin{code}[language=python, caption=BestTestCase.py, firstnumber=60]
def get(self, identifier):
    response = self._get(path=f"/{self.path}/{identifier}")
    self.assertTrue(response.status_code == HTTPStatus.OK)
    return response
\end{code}

I get metoden bliver der, ligesom i patch metoden, oprettet et respons objekt. Her bliver det dog oprettet ved at kalde \_get metoden. Efter oprettelsen bliver den, ligesom de andre eksempler, testet for dens status\_code, der skal være \texttt{OK} for at testen lykkes.\\

Delete metoden bliver kaldt som den sidste i TetstUser test\_all metoden, og tager identifier objektet som paramter. Først bliver response objektet oprettet ved at kalde \_delete metoden, der returnere et response objekt. Herefter tester assertTrue metoden om reponse status\_code er lig med no\_content. De pågældene metoder kan ses nedenfor: \\

\begin{code}[language=python, caption=UsersTest.py, firstnumber=17]
self.delete(identifier=identifier)
\end{code}

\begin{code}[language=python, caption=BestCaseTest.py, firstnumber=80]
def delete(self, identifier: str):
    response = self._delete(f"/{self.path}/{identifier}")
    self.assertTrue(response.status_code == HTTPStatus.NO_CONTENT)
    return response
\end{code}


\newpage
{\large\textbf{Testning EPG-Poller}}

\myparagraph{Test af HttpManger klassen}
HttpMangerTest klassen importerer klasserne Test, LoadConfig, CreditoroChannel, CreditoroProduction, Assertions.*, LocalData, Arrays.
Disse bliver brugt til at Checke om HttpManger Kan give de rigtige resultater.

\begin{code}[language=java, caption=HttpManagerTest.java, firstnumber=1]
package dk.creditoro.epg_poller.networking;

import org.junit.jupiter.api.Test;

import dk.creditoro.epg_poller.core.LoadConfig;
import dk.creditoro.epg_poller.models.CreditoroChannel;
import dk.creditoro.epg_poller.models.CreditoroProduction;

import static org.junit.jupiter.api.Assertions.*;

import java.time.LocalDate;
import java.util.Arrays;

class HttpManagerTest \{
    private HttpManager httpManager;
    private static String USER = LoadConfig.getLoadconfig().getUser();
    private static String PASSWORD = LoadConfig.getLoadconfig()
        .getPassword();
    public HttpManagerTest() {
        httpManager = new HttpManager();
    }
\end{code}

Her bliver der sat USER og PASSWORD, som skal bruges til at logge ind på api'et. LoadConfig står for at hente miljø variablerne.
Ved at gøre det denne måde kan, man altid skifte i hvordan den læser passwords.
Der efter bliver httpManager initialiseret.


\begin{lstlisting}[language=java, caption=HttpManagerTest.java, firstnumber=23]
    @Test
    void getTvTidChannels() {
        var channels = httpManager.getTvTidChannels();
        assertNotNull(channels);
        var tvTidChannels = channels.getChannels();
        assertNotNull(tvTidChannels);
        assertTrue(tvTidChannels.length > 0);
    }
\end{lstlisting}
Her bliver det testet om den kan få kandler ned fra tvtid.dk
Der bliver testet om den ikke er null. 
Checker om den kan give en liste tilbage på linje 28.
Checker om længen er størrer end null på linje 29.
Den ud pakker der bliver brugt, pakker null ud, hvis json object skifter. 
Der ved man at hvis tvtids data skifter eller man laver en opdatering der gør det ikke virker mere ville denne her test gøre det synligt.


\begin{lstlisting}[language=java, caption=HttpManagerTest.java, firstnumber=32]
    @Test
    void postChannel() {
        httpManager.login(USER, PASSWORD);
        //Create the channel that we are testing
        var channel = new CreditoroChannel("testChannel", "testURL");
        // Makes sure to delte the channel before we test if it can post.
        var getRespone = httpManager.getChannels(channel.getName());
        if (getRespone.length != 0){
            httpManager.deleteChannel( getRespone[0].getIdentifier() );
        }
        // Post the first channel
        var responeChannel = httpManager.postChannel(channel);
        assertEquals(channel.getName(), responeChannel.getName(), "Makes sure it it gets the same channels as it posted");
        // post the samme channel again
        var dublicatedResponeChannel = httpManager.postChannel(channel);
        assertNotEquals(channel, dublicatedResponeChannel, "Makes sure it can't post dublicate");
        // Delte the channel 
        getRespone = httpManager.getChannels(channel.getName());
        var httpRespone = httpManager.deleteChannel( getRespone[0].getIdentifier() );
        assertEquals(204, httpRespone, "Checks if we can delte channel");
    }
   
\}
Her bliver metoden postChannel på httpMangeren testet.

\end{lstlisting}