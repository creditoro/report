\documentclass[a4paper,12pt]{article}
\usepackage[utf8]{inputenc}
% Resten af pakkerne
\usepackage[english, danish]{babel}
\usepackage{csquotes}
\usepackage{float}
\usepackage{flafter}
\usepackage{graphicx}
\usepackage{setspace}
\usepackage{enumitem}
\usepackage{multirow}
\usepackage{lmodern}
\usepackage{amssymb,amsmath}
\usepackage{ifxetex,ifluatex}
\usepackage{lastpage} % Bruges i customtitlepage til at tælle sider
\usepackage[font=small,labelfont=bf]{caption}
\usepackage{lscape}
\usepackage{xargs}
\usepackage{tabularx}
\usepackage{comment}
\usepackage{pdfpages}
\usepackage{xcolor}

%\usepackage{minted}  <-- Fix me please :-)
\usepackage{listings}

% defines \RaggedRight
\usepackage{ragged2e}

\definecolor{mygreen}{rgb}{0,0.6,0}

\lstnewenvironment{code}[1][]%
{
   \noindent
   \minipage{\linewidth} 
   \vspace{0.5\baselineskip}
   \lstset{language=java, basicstyle=\ttfamily\footnotesize,frame=single,#1}}
{\endminipage}


\lstset{
    stringstyle=\color{mygreen},
    frame=single,                           % adds a frame around the code
    language=Java,                          % the language of the code
    breaklines=true,                        % sets automatic line breaking
    basicstyle=\small,                      % the size of the fonts that are used for the code
    keywordstyle=\color{blue},              % keyword style
    morekeywords={var, UUID, ROLE, EMAIL},  % if you want to add more keywords to the set
    numbers=left,                           % % where to put the line-numbers; possible values are (none, left, right)
    numberstyle=\small,                     % the style that is used for the line-numbers
    commentstyle=\color{mygreen},           % comment style
    captionpos=b                            % sets the caption-position to bottom
}
% Use for lstlisting formation caption
% \captionsetup[lstlisting]{ format=listing, labelfont=white, textfont=white, singlelinecheck=false, margin=0pt, font={bf,footnotesize}}

% \usepackage[
% backend=biber,
% style=alphabetic,
% sorting=ynt
% ]{biblatex}
% \addbibresource{appendices/bibliography.bib}

% \title{Bibliography management: \texttt{biblatex} package}
% \author{Overleaf}
% \date{ }

\usepackage{ifthen}

\usepackage{xcolor}
% \usepackage{csvsimple}
\usepackage{longtable}

% Margin
\usepackage{geometry}
\geometry{a4paper,  total={170mm,250mm},
 left=20mm,
 top=25mm}


% New Commands
\newcommand\myworries[1]{\textcolor{red}{#1}} 
\newcommand{\myparagraph}[1]{\paragraph{#1}\mbox{}\\}


% Needs to be the last package included
\usepackage{hyperref}
\hypersetup{
    colorlinks=true,
    linkcolor=black,
    filecolor=magenta,      
    urlcolor=blue,
    citecolor=black,
}


% Linjemellemrum
% \linespread{1.4}
\graphicspath{{./figures/}{../figures/}}

\setlength\parindent{0pt} % Removes indent

\begin{document}
    % Fjerner side tal
    \pagenumbering{gobble}
    \renewcommand{\thesection}{\Roman{section}} 
    \renewcommand{\thesubsection}{\thesection.\Roman{subsection}}
    % Forside
    \input{tex/i_forside}
    \newpage
    
    % Titelblad 
    \begin{tabular}{@{}l l} 
\textbf{Title:} & Creditoro \\
& \\
\textbf{Institution:} & Syddansk Universitet \\
& Det Tekniske Fakultet, Mærsk Mc-Kinney Møller Instituttet \\
& Campusvej 55, 5230 Odense M \\
& \\
\textbf{Uddannelse:} & Softwareteknologi \\
& \\
\textbf{Semester:} & 2. Semester \\
& \\
\textbf{Semestertema:} & Udvikling af cyber-physical softwaresystemer \\
& \\
\textbf{Kursuskode:} & ST2-PRO \\
& \\
\textbf{Projektperiode:} &  01.01.2020 - 29.05.2020\\
& \\
\textbf{ECTS:} & 10 ECTS\\
& \\
\textbf{Vejleder:} & Henrik Lykkegaard Larsen\\
& \\
\textbf{Projektgruppe:} & 06\\
& \\

\\
\end{tabular}

% VARIABLES
\newcounter{PROD}
\setcounter{PROD} {0}

%%%%


% Jakob Jakob Jakob Jakob Jakob Jakob Jakob Jakob Jakob Jakob Jakob
\ifnum \value{PROD}=1
    \includegraphics[scale=0.07]{figures/signatures/signatureJR.jpg}
    \vspace{-9.5mm}
\fi
\par\rule{\textwidth}{0.4pt}

Jakob Rasmussen, jakra19@student.sdu.dk\\
% -end- -end- -end -end -end- -end- -end -end -end- -end- -end -end -end-

% Kenneth Kenneth Kenneth Kenneth Kenneth Kenneth Kenneth Kenneth Kenneth

\ifnum \value{PROD}=1
    \includegraphics[scale=0.3]{figures/signatures/signature_kechr19.PNG}
    \vspace{-5mm}
\fi
\par\rule{\textwidth}{0.4pt}

Kenneth M. Christiansen, kechr19@student.sdu.dk\\
\vspace{3.5mm}
% -end- -end- -end -end -end- -end- -end -end -end- -end- -end -end -end-

% KEVIN KEVIN KEVIN KEVIN KEVIN KEVIN Kevin Kevin Kevin Kevin Kevin Kevin
\vspace{-6.5mm}

\ifnum \value{PROD}=1
    \includegraphics[scale=0.3]{figures/signatures/signature_kepet19.png}
    \vspace{-8mm}
\fi
\par\rule{\textwidth}{0.4pt}

Kevin K. M. Petersen, kepet19@student.sdu.dk
% -end- -end- -end -end -end- -end- -end -end -end- -end- -end -end -end-

% Kristian Kristian Kristian Kristian Kristian Kristian Kristian

\ifnum \value{PROD}=1
    \includegraphics[scale=0.04]{figures/signatures/signature_kjako19.jpg}
    \vspace{-9.5mm}
\fi
\par\rule{\textwidth}{0.4pt}

Kristian N. Jakobsen, kjako19@student.sdu.dk\\
% -end- -end- -end -end -end- -end- -end -end -end- -end- -end -end -end-

% Mathias Mathias Mathias Mathias Mathias Mathias Mathias Mathias Mathias

\ifnum \value{PROD}=1
    \includegraphics[scale=0.120]{figures/signatures/Signatur_mara816.png}
    \vspace{-5mm}
\fi
\par\rule{\textwidth}{0.4pt}

Mathias N. Rasmussen, mara816@student.sdu.dk\\
% -end- -end- -end -end -end- -end- -end -end -end- -end- -end -end -end-

% Simon Simon Simon Simon Simon Simon Simon Simon Simon Simon Simon Simon

\ifnum\value{PROD}=1
    \includegraphics[scale=0.042]{figures/signatures/signatureSJ.png}
    \vspace{-3.5mm}
\fi
\par\rule{\textwidth}{0.4pt}

Simon Jørgensen, sijo819@student.sdu.dk\\
% -end- -end- -end -end -end- -end- -end -end -end- -end- -end -end -end-


%Bottom of page
%\vfill


\begin{tabular}{@{}l l}
Antal sider:    & \myworries{sider} \\ % \pageref{LastPage} virker ikke mere, da der ikke er sidetal på bilag
Bilag:          & 1 bilag 
\end{tabular}

\vspace{3.5mm}

\begin{footnotesize}

\textbf{Ved at underskrive dette dokument bekræfter hvert enkelt gruppemedlem, at alle
har deltaget lige i projektarbejdet, og at alle således hæfter kollektivt for rapportens indhold.}
\end{footnotesize}
    \newpage

    % Resume
    \section{Resume}
Denne sektion skal indeholde:

\begin{itemize}
    \item Hovedresultater og konklusioner  – hvad kom der ud af arbejdet
\end{itemize}{}

I dette projekt bliver der arbejdet ud fra TV2's problemstilling omhandlende muligheden for at flytte krediteringer fra TV til en anden platform. \myworries{Denne plads kan så udfyldes med andet som fx. reklamer og promoveringer for egne programmer.} Gruppen har valgt at bygge et fuldt funktionelt program, inklusive Rest API og EPG poller. Dertil er gruppen fundet frem til en problemformulering der omhandler udviklingen af et krediteringssystem, hvordan krediteringerne skal gøres tilgængelige samt håndteres. Derudover undersøges det hvordan systemet kan indeholde krediteringerne.
Projektgruppen har valgt at afgrænse projektet ved at lave en prototype til et færdigt program. Ideelt ville systemet også have en webside, dog er dette vurderet til værende udenfor projektets scope.

Overordnet benyttes UP (Unified Process) i hele projektet. UP opdeler hele projektet i 4 faser: \textit{Inceptionsfasen} hvor vigtige krav og kristiske risici identificeres. \textit{Elaborationfasen} hvor en iterativ udvikling af krav, design, analyse og test konstrueres ud fra overordnede kravspecifikationer. \textit{Konstruktionfasen} hvor systemet konstrueres - den faktiske kodeudformning. \textit{Overgangsfasen} hvor det undersøges om systemet er færdigt og ikke har mangler.

I starten af projektets analysefase er et brugsmønsterdiagram, herunder aktørliste og brugs-mønsterliste, udformet. Dette er med til at give et billede af hvordan systemet skal bygges op, samt hvilke aktører der skal kunne interagere med hvilke brugsmønstre. Til prioritering af brugsmønstrene er der foretaget en MoSCoW-analyse.
Til identificering af ikke-funktionelle krav er der brugt metoden FURPS. Her er der sat krav op der omhandler Functionality, Usability, Reliability, Performance og Supportability. For yderligere specificering af ikke funktionelle krav er FURPS+ benyttet - FURPS med ekstra specificerende kategorier med til at opfylde kundens behov.

Til projektstyring benyttes Scrum til at fordele og få overblik over de forskellige opgaver, der opdeles i mindre issues. De forskellige arbejdsopgaver inddeles i sprints som forløber sig over en given periode - i dette projekts tilfælde to uger.

\myworries{Hovedresultater og konklusioner}



    \newpage
    
    % Forord
    \section{Forord}

Denne rapport er skrevet i forbindelse med 2. semesterprojekt i Softwareteknologi på Syddansk Universitet i samarbejde med TV2, der ønsker at lave et krediteringssystem der skal flytte krediteringer fra TV til en anden platform. Denne plads kan så erstattes af reklamer og promoveringer for andre programmer.\\
Rapporten beskriver gruppens arbejdsform og fremgangsmåde for projektet, samt de konklusioner gruppen er kommet frem til. Hensigten med rapporten er at dokumentere arbejdsprocessen henimod udviklingen af et system til håndtering af krediteringer.\\
En særlig tak skal lyde til vejleder Henrik Lykkegaard Larsen samt Troels Mortensen for hans videoserie på YouTube omhandlende MVVM (Model - View - View Model).\\


God fornøjelse med læsningen.\\

Jakob Rasmussen\\
Kenneth M. Christiansen\\
Kevin K. M. Petersen\\
Kristian N. Jakobsen\\
Mathias N. Rasmussen\\
Simon Jørgensen\\
    \newpage

    % Indholdsfortegnelse
    \setcounter{tocdepth}{2} % anything below subsection will not be added
    \input{tex/v_indholdsfortegnelse}
    \newpage
    
    % Læsevejledning
    \section{Læsevejledning}

Rapporten er disponeret sådan, at \textit{afsnit 1 - 3} beskriver projektets udgangspunkt. Herunder findes det essentielle problem rapporten er udformet ud fra, de benyttede metoder og værktøjer igennem hele projektet, planlægningen og strukturering samt det faglige vidensgrundlag.

\textit{Afsnit 5-6} indeholder kravspecifikationer herunder udarbejdelsen af de overordnede krav. Derudover kan de detaljerede krav findes her.

Analysen af projektet findes i \textit{afsnit 7 - Analyse}, hvor der bliver udformet en brugsmønsteranalyse og brugsmønsterrealisering samt et resumé af konstrueringen af EAST-ADL modeller ud fra funktionelle og ikke funktionelle krav.

I \textit{Afsnit 8-9} kan der læses om projekts designfase. Herunder findes systemets design samt systemets subsystemer. Det er her også muligt at se designafsnittet for databasen.

I \textit{afsnit 10 - Implementering} kan der læses om konverteringen fra design til kode, samt vigtige implementeringsbeslutninger. Dette afsnit dækker hele systemet inklusiv subsystemer.

\textit{Afsnit 11 - Test} indeholder tests af en række udvalgte metoder og klasser fra systemet i henholdsvis 1. og 2. iteration.

Rapporten rundes af med \textit{afsnit 12 - 15}, hvor diskussion, konklusion, perspektivering samt procesevaluering er at finde. I disse afsnit vil der blive fokuseret på, hvad der opnået i projektet, hvad der kan gøres bedre, opsummering af resultater, samt en evaluering over hele projektets proces.

Vejleder- og samarbejdsaftale vil være at finde i \textit{bilag \ref{vejlederaftale}} og \textit{\ref{samarbejdsaftale}}.

Kun udvalgte modeller, tabeller og diagrammer er indsat i rapporten, resten vil være at finde i \textit{bilag}.

Ønskes det at læses om systemets subsystemer - herunder Rest API og EPG Poller - vil dette være at finde i \textit{bilag}. Her findes analyse, design, implementering og tests af de to subsystemer.

    \newpage
    
    \section{Redaktionelt}
Denne sektion skal indeholde:

\begin{itemize}
    \item Skriveprocessen og ansvarsområder i skriveprocessen.
    \item Ansvarsområder kan fx beskrives på fx følgende form: (skema med afsnit, ansvarlig, bidrag af og kontrolleret af)
\end{itemize}{}

\begin{table}[H] % <---- Remember to change me :)
    \begin{tabularx}{\textwidth}{|p{7cm}|X|X|X|}
        \hline
        \textbf{Afsnit}                     &  \textbf{Ansvarlig}  & \textbf{Bidrag af} & \textbf{Kontrolleret af}\\
        \hline
        Forside                             & Kevin & Kenneth & tekst \\
        \hline
        Titelblad                           & Kevin & Kenneth & tekst \\
        \hline
        Resumé                              & tekst & tekst & tekst \\
        \hline
        Forord                              & tekst & tekst & tekst \\
        \hline
        Læsevejledning                      & tekst & tekst & tekst \\
        \hline
        Indledning                          & Kenneth & tekst & tekst \\
        \hline
        Metoder \& Værktøjer                & Kenneth & tekst & tekst \\
        \hline
        Planlægning                         & Jakob & tekst & tekst \\
        \hline
        Faglige Vidensgrundlag              & Kenneth & tekst & tekst \\
        \hline
        Overordnede Krav                    & Mathias & tekst & tekst \\
        \hline
        Detaljerede Krav                    & Mathias & tekst & tekst \\
        \hline
        Analyse                             & Jakob & tekst & tekst \\
        \hline
        Design                              & Kevin \& Simon & tekst & tekst \\
        \hline
        Databasedesign                      & Kristian & tekst & tekst \\
        \hline
        Implementering                      & Kristian \& Simon & tekst & test \\
        \hline
        Test                                & Kristian \& Simon & tekst & tekst \\
        \hline
        Diskussion                          & tekst & tekst & tekst \\
        \hline
        Konklusion                          & tekst & tekst & tekst \\
        \hline
        Perspektivering                     & tekst & tekst & tekst \\
        \hline
        Processevaluering                   & tekst & tekst & tekst \\
        \hline
        Bilag A - Oversigt Over Kildekode   & tekst & tekst & tekst \\
        \hline
        Bilag B - Brugervejledning          & tekst & tekst & tekst\\
        \hline
        Bilag C - Samarbejdsaftale          & Alle &  & Alle\\
        \hline
        Bilag D - Vejlederaftale            & Alle &  & Alle\\
        \hline
        Bilag E - Projektlog                & Kevin \& Kenneth &  & \\
        \hline
        Bilag F - Rapportkontrolskema       & tekst & tekst & tekst\\
        \hline
        Bilag G - Inceptionsdokument        & Alle &  & Alle \\
        \hline
        Bilag H - Andre Bilag & tekst       & tekst & tekst \\
        \hline
    \end{tabularx}
    \caption{Ansvarsområder i skriveprocessen}
    \label{tab:redaktionelt}
\end{table}
    \newpage
    
    % Start counting from this line
    \pagenumbering{arabic}
    \setcounter{page}{1}

    \renewcommand{\thesection}{\arabic{section}} 
    \renewcommand{\thesubsection}{\thesection.\arabic{subsection}}
    \setcounter{section}{0}
    % Introduktion
    \section{Indledning}
% ------------------------ Baggrund for projektet ------------------------------------------------
Når et program bliver broadcasted på en TV station skal krediteringer vises. Dette gøres i slutningen af programmet, i maksimalt 30 sekunder. Det betyder, at der ikke altid er tid til at vise alle krediteringer, og derfor prioriteres de før de vises. \\
Hvis de 30 sekunder for hvert program kunne frigøres, kan danske TV stationer bruge tiden på at vise noget andet, som f.eks. reklamer og promovering af eget indhold. Derved har TV 2 mulighed for at øge deres årlige indtægter med op til 60 millioner kroner. \\
TV 2 har brug for et system, der kan administrere krediteringer for programmer produceret i Danmark. Hertil skal der kunne tilføjes nye krediteringer i systemet for nye produktioner, samt det skal være muligt at kunne søge efter eksisterende krediteringer. Det skal være muligt at kunne se hvilken rolle en given person har haft i en produktion, da denne person kan have haft flere forskellige roller på flere forskellige produktioner.

% ------------------------ Projektets rammer og baggrunden for projektet -------------------------
\subsection{Projektrammer}
Denne sektion har til formål at opridse rammerne for projektet, samt hvilket område projektgruppen arbejder indenfor.

\subsubsection{Krav til Projektet}
Systemet skal så vidt muligt skrives i programmeringssproget Java. \\
Krediterings-data skal lagres i en database, og i dette projekt skal den brugte database være SQL baseret. Der skal bruges PostgreSQL.\\
Systemet forventes ikke at være et færdigt system, men en række forslag til løsninger der opfylder systembehovet. Forslagene skal inkludere:

% ------------------------- OPDATER DENNE TIDSPLAN ----------------------------------------------
\begin{itemize}
    \item Krav
    \item Analyse
    \item Design
    \item Implementering
    \item Test
\end{itemize}
% ----------------------------------------------------------------------------------------------

\noindent
Producere der kan tilføje og redigere i krediteringerne, skal kun have mulighed for at redigere i de produktioner, de selv ejer.\\
Det forventes at krediteringssystemet er kompatibelt med andre systemer (f.eks. fra Stofa, YouSee etc.).

\subsubsection{Tidsplan}
Tidsplanen har til formål at skabe overblik og styring over projektet. 
Den giver gruppemedlemmerne et overblik over, hvornår de forskellige dele af projektet skal starte og slutte, og derved bliver det hurtigt klart hvis tidsplanen skrider. \\

\begin{landscape}
    \begin{figure}
        \centering
        \includegraphics[scale=0.30]{figures/grantt_udvidet.png}
        \caption{Tidsplan for projektet}
        \label{fig:gantt}
    \end{figure}{}
\end{landscape}


% ------------------------ Resume af udleverede case ----------------------------------------------
\subsubsection{Igangsættende Problem}
TV2 ønsker at frigøre 30 sekunders krediterings tekster efter hvert program, så de i stedet kan bruge tiden på at vise reklamer. Problemet består i at disse krediterings tekster, så skal vises på en anden platform. I tabel \ref{table:kravFraCase} ses kravene fra TV2's projektcase: \\

\begin{table}[ht]
    \begin{tabularx}{\textwidth}{|p{10cm}|X|}
        \hline
        \textbf{Beskrivelse} & \textbf{Type} \\
        \hline
        “Vi har brug for  et krediterings system der kan  håndtere  dansk TV content” 
        & En vag opgave \\
        
        \hline
        "Dette inkluderer muligheden for at oprette nye krediteringer i systemet, når en ny produktion bliver lavet, samt at have mulighed for at søge efter en given produktion og få en liste af krediteringer, forbundet til denne. Det burde også være muligt at se hvilken rolle en given person har haft i en produktion, eftersom en person kan have flere forskellige roller i forskellige produktioner."
        & Ønske om en bestemt løsning \\
        
        \hline 
        "Producers/TV-stationer burde være i stand til at redigere krediteringer for programmer/produktioner de ejer. De burde også være i stand til at redigere disse produktioners ID. Systemadministratorer skal kunne vedligeholde (oprette, læse, opdatere og slette) personer, krediteringer og personer."
        & Ønske om en bestemt løsning \\
        
        \hline
        "Til slut skal systemet kunne offentliggøre en service som andre systemer kan bruge. Disse systemer kan f.eks. være en hjemmeside eller en applikation. Disse andre systemer skal også kunne bruge API'et, så data'et kan blive brugt i allerede eksisterende systemer (såsom TVTID.dk - TV 2's TV-Guide)."
        & Ønske om en bestemt løsning \\
        
        \hline
        "En form for adgangskontrol skal implementeres, til de beskyttede dele af systemet (oprettelse, opdatering, slettelse, osv. af data)
        & Ønske om en bestemt løsning \\
        
        \hline
        "Der skal være en offentligt tilgængelig del af systemet, hvor det er muligt at se krediteringer uden at logge ind."
        & Ønske om en bestemt løsning \\
        
        \hline
        “Nuværende løsning er begrænset til 30 sekunder, og dermed kan alle krediteringerne ikke altid vises i praksis” 
        & Et problem \\
        \hline
    \end{tabularx}    
    \caption{Krav fra TV2s projektcase}
    \label{table:kravFraCase}
\end{table}

% -----------------------------------------------------------------------------------------------------
\subsubsection{Identifikation af Problemet}
Som det er nu bliver krediteringer vist i slutningen af et program. Ifølge reglerne for visning af krediteringer, må krediteringer ikke vises mere end 20 sekunder for produktioner under 60 minutter, og 30 sekunder for produktioner over 60 minutter. Dette giver en del problemer. For det første betyder den begrænsede varighed, at ikke alle medarbejdere kan krediteres. Dette ender ud i at der skal prioriteres i krediteringerne, før de bliver vist på TV. Derved får alle medarbejdere ikke den anerkendelse de burde.
Hvis krediteringer flyttes til et eksternt system, og derved ikke bliver vist på TV, kan man undgå at skulle prioritere. Alle kan derved få den fortjente kredit. Derudover vil det også give mulighed for at vise noget andet, som f.eks. reklamer eller promoveringer for andre programmer (red: eget indhold).
\footnote{\href{https://www.dr.dk/NR/rdonlyres/00221a7b/dpikscstjptklixxdnjgywgeuakhwpog/DR_kreditmanual_050810.pdf}{DR’s krediteringsregler for TV}}
% brug bibliography. show me the way

\noindent
Et sådan eksternt system vil også hjælpe med oprettelsen af nye krediteringer, ved at gøre processen hurtigere og nemmere, samt mere overskueligt. Dertil har gruppen valgt at arbejde med samtlige/alle problemstillinger givet af TV2 i tabel \ref{table:kravFraCase}, og lave en prototype til et funktionsdygtigt system. Angående valget med at arbejde med samtlige problemstillinger præsenteret af TV2, har gruppen konkluderet det som værende realistisk jævnfør figur \ref{fig:gantt}. Denne prototype vil kunne bruges som et udkast til et endeligt system.


% ------------------------ Mål og Formål med projektet ----------------------------------------------

\subsection{Formål med rapporten}
Formålet med inceptionsfasen er at fastlægge systemets omfang, der bliver udformet en overordnet kravspecifikation, kravene prioriteres og metoderne i elaborationsfasen beskrives. Dette sker gennem en nærmere undersøgelse af problemstillingen, indsamling af information og under kundemøder hvor kravene indsamles (eliciteres).\\

\noindent
Målene for inceptionsfasen kan således opstilles i punktform:
\begin{itemize}
    \item At gennemføre kravudvikling
    \item At identificere kritiske risici
    \item At fastlægge projektets metoder i elaborationsfasen
\end{itemize}

\subsection{Problemanalyse}
% PRE review: Hvordan kan vi udvikle et samlet krediteringssystem, der giver mulighed for at erstatte rulletekster efter et endt program?


% ------------------------ Problemformulering & Afgrænsning ------------------------------------------
\subsection{Problemformulering \& Afgrænsning}
\textit{Hvordan kan vi udvikle et samlet krediteringssystem, der giver mulighed for at erstatte rulletekster efter et endt program?}

\begin{enumerate}
    \item Hvem skal kunne håndtere krediteringer?
    \item Hvordan skal krediteringerne gøres tilgængelige, og hvordan skal seerne refereres dertil?
    \item Hvordan kan man oprette et system som kan indeholde krediteringer?
\end{enumerate}

\noindent % afgrænsning
Projektgruppen har valgt at afgrænse dette projekt, ved at konstruere en prototype til et system.
% Indsat post review. 

\noindent % Det vi har med:
Projektgruppen har valgt at lave et system der ligger tæt op ad det oprindelige foreslag fra TV2s projektcase. Dette indebærer alle kravene i tabel \ref{table:kravFraCase}.
\begin{figure}[H]
\centering
\includegraphics[scale=0.3]{figures/tv2_system.png}
\label{fig:tv2_system}
\caption{Foreslag til systemtegning - © TV2}
\end{figure}
% Indsæt figur af deres oprindelige systemtegning?

\noindent % Det vi vælger fra:
I et produktionsklart system vil det være ideelt at have en webside, men dette har vi konkluderet som værende uden for projektet.
    \newpage
    
    % Metoder & Værktøjer
    \section{Metoder \& Værktøjer}
Dette afsnit har til formål at beskrive de metoder og værktøjer der er blevet brugt i løbet af udviklingen af produktet og udarbejdelsen af rapporten. 

% ------------------------------ Metoder --------------------------------------
\subsection{Metoder}
%% ----------------------------- Brugsmønsterdiagram --------------------------
\subsubsection{Brugsmønsterdiagram}
Det første skridt i udviklingen af et brugsmønster er at finde og definere de forskellige aktører der vil interagere med systemet. En aktør kan defineres som alt der kommunikerer med systemet og ikke selv er en del af systemet. Et eksempel på dette kunne være en kunde på en webshop. Disse aktører opstilles i en tabel sammen med de brugsmønstre hver aktør kan tilgå. \\


Da kravindsamling er en evolutionær aktivitet, bliver alle aktører ikke nødvendigvis identificeret i første iteration. Det er muligt at identificere primærer aktører i løbet af første iteration, og først senere i forløbet blive i stand til at identificere sekundære aktører, når man får mere viden om systemet. \textit{Primære} aktører interagerer med systemet for at opnå påkrævede systemfunktioner, og ud fra det, få noget ud af at bruge systemet. \textit{Sekundære} aktører støtter systemet så de primærer aktører kan gøre deres arbejde. Når aktørerne er fundet kan brugsmønstrene findes. Et brugsmønster angiver et scenarie en aktør kan interagere med. \\


Når både aktører og brugsmønstre er fundet, kan man opstille et brugsmønsterdiagram for at give en visuel forståelse for hvilke aktører der kan tilgå hvilke brugsmønstre. Et eksempel på et brugsmønsterdiagram kan ses på figur \ref{fig:usecasemodel} \\


Brugsmønstermodellen hjælper udvikleren med at forstå brugeren, så systemet kan konstrueres. Idet der er et samlet billede af hvordan systemet skal se ud, vil udviklingen være mere målfast da kravene og deres forhold til brugeren er klare.


%% ----------------------------- FURPS+ ---------------------------------------
\subsubsection{FURPS+}
FURPS er en model til klassificering af ikke-funktionelle krav, og er med til at give en detaljeret beskrivelse af kravene. Akronymet står for:

\begin{description}
    \item [Functionality:] Hvad kunden vil have. Dette inkluderer også sikkerhedsforanstaltninger.
    \item [Usability:] Hvor effektivt er produktet fra brugerens synspunkt? Er produktet æstetisk acceptabelt? Er dokumentationen fyldestgørende? 
    \item [Reliability:] Hvad er den mest acceptable system nedetid? Er systemfejl forudsigelige? Er det muligt at demonstrere hvor præcise resultaterne er? Hvordan bliver systemet gendannet?
    \item [Performance:] Hvor hurtigt skal systemet være? Hvad er den maksimale responstid? Hvad er gennemløbet? Hvor meget hukommelse bruger systemet?
    \item [Supportability:] Kan systemet testes? Er det muligt at konfigurere systemet, udvide det, installere det, og yde service på systemet.
\end{description}


+-tegnet står for supplerende behov kunden kan have, og omfatter:

\begin{description}
    \item [Design constraints:] Har I/O enheder eller database management systemer indflydelse på hvordan softwaren skal opbygges?
    \item [Implementation requirements:] Er det nogle standarder programmørerne skal overholde? er testdrevet udvikling nødvendigt? 
    \item [Interface requirements:] Hvilke downstream feeds skal der laves? Hvilke andre systemer skal systemet samarbejde med?
    \item [Physical requirements:] Hvilken hardware skal systemet implementeres på?
\end{description}


%% ----------------------------- MoSCoW --------------------------------------
\subsubsection{MoSCoW}
MoSCoW er en vigtig prioriteringsmodel indenfor softwareudvikling, da den beskriver hvilke dele af systemet der som minimum skal laves før produktet kan accepteres. De brugsmønstre, udvilkerne af systemet skal tage udgangspunkt i, bliver prioriteret med kunden, så det bliver klart hvilke dele af systemet der skal udvikles først. Disse bliver opstillet i en MoSCoWmodel, så der let skabes et overblik. MoSCow er et akronym der står for:

\begin{description}
    \item [Must have] - betyder skal have og er det som der minimum skal være med for at softwaren virker og kan accepteres af kunden. 
    \item [Should have] - betyder det som burde være med og det kunden gerne vil have med.
    \item [Could have] - betyder det som kunne være med, hvis der er tid nok. 
    \item [Won't have (this time)] - betyder det som der ikke skal prioriteres nu, men måske i en anden iteration.
\end{description}


%% ----------------------------- UP & Scrum ----------------------------------
\subsubsection{UP \& Scrum}{\label{subsubsection: up_scrum}}

\myparagraph{UP}
'UP' er en forkortelse af 'unified process' og består af fire faser. De fire faser er beskrevet i den rækkefølge som de udføres. \\


\textbf{Inceptionsfasen} er der for at finde ud af om projekt overhovedet kan gennemføres, bestemme hvilket anvendelseområde systemet har og identificere vigtige krav og kritiske risici. \\


\textbf{Elaborationfasen} er for at lave en iterativ udvikling af de forskellige krav, design, analyse og test ud fra den overordnede kravspecifikation og den prioritering der er lavet. I Elaborationfasen vil der blive brugt Scrum. \\


\textbf{Konstruktionfasen}, hvor fokus vil være på udvikling af komponenter og andre funktioner.
I fasen vil der blive brugt UML til at identificere hvilke klasser og komponenter der skal være. Det er i denne fase kodeningen kommer til at ske og den første iteration af software produktet.  \\


\textbf{Overgangfasen}, hvor der vil være fokus på at få et færdigt softwareprodukt.
Det vil man gøre ved, at se om man har implementeret de aftalte funktioner og i dialog med kunden finde ud af om de er tilfredse med produktet.

\myparagraph{Scrum} 
Scrum er en agil udvilkingsmetode der benyttes til at lede og kontrollere leverancer af produkter. Scrum består af 3 faser: 

\begin{enumerate}
    \item Forberedelse
    \item Eksekvering
    \item Idriftsættelse
\end{enumerate}


I forberedelsesfasen er der 3 aktiviteter: Produktvision, der er en overordnet beskrivelse af løsningen og dens omfang, product roadmap, der er en overordnet plan for hvornår vigtige funktioner forventes leveret, release plan, der er en inddeling af product roadmap i en eller flere udgivelser, hvor den første udgivelser består af de minimumsfunktioner der er beskrevet i MoSCoW modellen.\\


I eksekveringsfasen er der tre artefakter og tre ceremonier. Artefakterne består af product backlog, der er en samling over alle krav for systemet, sprint backlog, der består af de krav gruppen skal implementere i den kommende sprint, burndown chart, der er en visualisering af gruppens fremskridt. Ceremonierne består af sprint planlægning, hvor planlægningen af de enkelte sprints finder sted, dagligt stand up, der er et dagligt koordineringsmøde hvor gruppemedlemmerne snakker om hvad de har lavet siden sidste møde og hvad der skal laves til næste møde, sprint review, der er en gennemgang af hvordan sprintet er forløbet og bruges til at lave eventuelle rettelse i planlægningen af næste sprint. \\


I idriftsættelsesfasen bliver sprintet idrisftsat, hvilket betyder at der bliver frigivet til kunden i form af et nyt softwareprodukt eller en opdatering til en eksisterende produkt.\\


Scrum vil blive brugt i elaborationsfasen til at nedbryde de krav der blev defineret i inceptionsfasen. Der vil blive benyttet en sprintperiode på 2 uger, og da sprintperioden er kort (normalt bruges 1-4 uger) er det vigtigt at kravene bliver brudt ned til User Stories der kan nåes indenfor 2 sprints.
Gruppen vil i projektet benytte værktøjet ZenHub til GitHub for at integrere Scrum ind i projektet. Dette giver mulighed for at samle projekt management og kode på gruppens GitHub siden (\url{https://github.com/creditoro}).

Til projektet bliver der benyttet et scrum board med følgende kolonner:\\

\begin{table}[ht]
    \begin{tabularx}{\textwidth}{|>{\RaggedRight}X|>{\RaggedRight}X|>{\RaggedRight}X|>{\RaggedRight}X|>{\RaggedRight}X|>{\RaggedRight}X|}
    \hline
    \textbf{New Issues} & \textbf{Icebox} & \textbf{Backlog} & \textbf{In Progress} & \textbf{Done} & \textbf{Closed} \\ \hline
     & Issues med lav prioritet & Kommende issues & Igangværende issues & Færdige issues der bliver lukket næste sprint møde & \\ \hline
    \end{tabularx}
    \caption{Scrum Board}
    \label{tab:scrumboard}
\end{table} 

Alle issues er sorteret fra top til bund alt efter prioritet. \\

\newpage
% ------------------------------ Værktøjer -----------------------------------
\subsection{Værktøjer}
I tabel \ref{tab:tools} ses de værktøjer projektgruppen har benyttet under inceptionsfasen og de værktøjer, der skal bruges fremadrettet.
\begin{table}[ht]
    \begin{tabularx}{\textwidth}{|>{\RaggedRight}p{4cm}|>{\RaggedRight}X|}
        \hline
        \textbf{Værktøj} & \textbf{Beskrivelse} \\
        \hline
        PostgreSQL          &   PostgreSQL er en open-source objekt-relationel database server.\\
        \hline
        \multirow{3}{*}{GitHub}              &   GitHub er en web-baseretkollaborations platform henvendt til software udviklere, der gør det muligt at versions-kontrollere projekter.\\ 
        \hline
        \multirow{2}{*}{Overleaf}            &   Overleaf er en online skriveplatoform for \textbf{LaTeX}, hvor man kan være flere brugere der skriver samtidig. \\
        \hline
        UML                 &   Unified Modeling Language \\
        \hline
        \multirow{2}{*}{IntelliJ}            &   Integreret udviklings miljø, som primært bruges af gruppens medlemmer. \\
        \hline
        \multirow{2}{*}{ZenHub}              &   ZenHub er en platform der gør det lettere at anvende Scrum i praksis.  \\
        \hline
        \multirow{2}{*}{Scrum Board}         &   Et Scrum Board er et værktøj, der har til formål at gøre opgarverne i Sprint og Backlog synlige og overskuelige.\\
        \hline
        \multirow{2}{*}{Pair Programming}    &   Pair programming er en softwareudvilkingsteknik, hvor to programmører arbejder sammen ved én computer.\\
        \hline
        \multirow{3}{*}{Klassediagram}       &   Bruges til visuelt at vise hvordan softwaresystemer er opbygget. I diagrammet beskrives systemets klasser, metoder og værdier klassen indeholder, samt klassernes relationer til hinanden.\\
        \hline
        SonarCloud          &   Online service til scanne kode for bugs, vulnerbilities og code smells.\\
        \hline
    \end{tabularx}
    \caption{Værktøjer til projektarbejdet}
    \label{tab:tools}
\end{table}
    \newpage
    
    % Planlægning
    \section{Planlægning}
Dette afsnit har til formål at fremvise gruppen planlægning af eleborationsfasen sammenlignet med det faktiske arbejde, samt komme nærmere ind på hvordan gruppen har planlagt arbejdet i de enkelte sprints og iterationer.

\subsection{Plan for Elaborationsfasen og Det Faktiske Udviklingsarbejde}
\begin{itemize}
    \item Plan For elaborationsfasen og de enkelte iterationer. Brug af prioritering af krav i planlægningen.
    \item Det faktiske udviklingsarbejde. Faserne, iterationerne og det faktiske  arbejde i dem? \\
\end{itemize}

\noindent
Semesterprojektet er delt op i 2 iterationer. Til første iteration er planen at få sat REST API'et og databasen op. Disse skal indeholde vores brugere, kanaler, produktioner, personer og krediteringer så det kan hentes via API'et, som skrives i python og databasen er en relationel database i PostgreSQL. For alle API'ets moduler skal metoderne POST, GET, PATCH, PUT, DELTE og UPDATE implementeres. Disse metoder skal kunne tilgås via koden, så vi nemt og hurtigt kan hente dataet.\\
Til desktop-klienten skal vi i første iteration lave design mockup, til en del af programmet, så vi har en ide til hvordan programmet skal se ud. Her skal vi lave fxml dokumenterne for login-siden, gennemse kanaler og gennemse produktioner. Vi skal også implementere login, så brugerne kan logge ind via API'et. På siden for "gennemse kanaler"  skal vi hente data fra TVtid.dk via en EPG Poller. Derved kan vi se hvilke kanaler vi skal vise i vores program. \\
\myworries{Browse Channels Datamodel} 

\noindent
I anden iteration implementere roller i vores system. Indtil videre har målet være at vise programmet fra systemadministratorens synspunkt, men i anden iteration er planen at brugere nu skal have forskellige roller. Dette indebærer at ikke alle brugere ser samme knapper og information som f.eks. systemadministratoren. \myworries{Systemadministratoren skal bl.a. have mulighed for at oprette nye kanaler, hvor kanaladministratoren ikke skal.}
% Skal vi overhoved oprette nye kanaler? Vi henter dem jo bare fra vore API?
Vi skal lave design mockup til resten af programmet, bl.a. Produktions-siden og kanal siden. Derudover skal vi, til anden iteration, kunne trykke på kanalerne og se hvilke produktioner der hører til. Vi skal dertil også kunne importere nye produktioner fra TVtid via vores EPG Poller, samt oprette nye produktioner og nye krediteringer hertil.\\
\myworries{Channel Model \& Channel ViewModel}

\subsection{Backlogs}
Her kommer vores backlogs fra Zenhub til at blive præsenteret.

\subsection{Rollefordeling i projektgruppen}
Her kommer rollefordelingen i projektgruppen til at blive præsenteret.\\

\noindent
Der bliver taget udgangspunkt i Belbins rolleprofil, hvor det bliver sammenlignet med den faktiske rollefordeling.

\subsection{Ceremonier}
Her bliver Sprintplanlægning og dagligt stand up præsenteret. Sprint review kommer ikke til at blive nævnt, da dette er en af vores scrum-buts.

\subsection{Scrum-buts}
Sprint review bl.a.
    \newpage
    
    % Faglige vidensgrundlag
    \section{Faglige Vidensgrundlag}
\myworries{Begrebsdefinitioner, teori og  fagligt vidensgrundlag} \\
\noindent

Dette afsnit har til formål at dækker over den faglige viden gruppen skal have, for at kunne udføre projektet.

\subsection{JAVA}
At have kendskab til Java er en vigtig forudsætning for udarbejdelsen af projektet. Systemet vil hovedsageligt blive programmeret i sproget Java. Det faglige niveau svarer til 2. semesterstuderende på Softwareteknologi. Dette indebærer blandt andet forståelse af JavaFX og Scenebuilder.
Arbejdet med Java i projektet forudsætter derudover forståelse for basale programmeringsprincipper og forståelse for det objektorienterede programmeringsparadigme. 

\subsection{Python}
Et grundlæggende kendskab til Python kræves for at kunne forstå samt implementere REST Api'et.

\subsection{Database}
Systemet vil indeholde en lang række data, som skal lagres i en database. Det er derfor nødvendigt at have forståelse for databaser, databasestrukturer, relationelle SQL-databaser og SQL-queries. Databasen der vil blive benyttet i projektet er PostgreSQL, en basal viden om databasesproget/programmeringssproget SQL er derfor nødvendig. 
Al nødvendig viden er givet i SDU's 'Data Management' kursets pensum.

\subsection{Ubuntu \& Docker}
En basal forståelse for Ubuntu (eller andet Linux baseret distro) og Docker kræves for opsætning af REST Api'et og databasen.

\subsection{Relevante Eksisterende Løsninger}
\textbf{IMDb} \\
IMDb (Internet Movie Database) er en online database bestående af film, serier, medvirkende m.m. Man har mulighed for at søge efter informationer ved at referere til blandt andet førnævnte titler. IMDb har også et ratingsystem, der gør det muligt at bedømme film, serier etc.\\
\textbf{Rotten Tomatoes} \\
Rotten Tomatoes er på lige fod med IMDb, en database for film, serier, medvirkende m.m. Man kan på Rotten Tomatoes også søge information. Rotten Tomatoes distancerer sig fra IMDb, ved både at tage ratings fra sine brugere og et panel af anmeldere. 
    \newpage
   
    % Hovedtekst
    \section{Overordnede Krav}
Denne sektion skal indeholde:

\begin{itemize}
    \item Opdateret resume af overordnede krav fra inceptionsdokument
    \item Inklusiv overordnet brugsmønsterdiagram og oversigt over supplerende krav
\end{itemize}{}

% ---------------------------- Overordnede krav ------------------------------------
\subsection{Resume af Overordnede Krav}

\begin{table}[ht]
    \begin{tabularx}{\textwidth}{|p{1cm}|p{4cm}|X|}
        \hline
        \textbf{ID} & \textbf{Navn} & \textbf{Beskrivelse} \\
        \hline
        K01 & Brugergrænseflade & Systemet skal tilgås via en dansk brugergrænseflade \\
        \hline
        K02 & Brugerroller & Systemet skal indeholde brugerroller \\
        \hline
        \label{K03}K03 & Tildel roller & Kanaladministrator skal kunne tildele producer- og kanaladministrator roller \\
        \hline
        K04 & Slet bruger & Systemadministratoren skal kunne slette brugere \\
        \hline
        K05 & Se krediteringer & Alle skal kunne se krediteringer \\
        \hline
        K06 & Søg efter krediteringer & Alle skal kunne søge efter og se krediteringer for alle programmer \\
        \hline
        K07 & Opret krediteringer & Specielle brugere, kanaladministratore og systemadmin skal kunne oprette
        krediteringer for et givent program \\
        \hline
        K08 & Rediger krediteringer & Specielle brugere, kanaladmin og systemadmin skal kunne redigere krediteringer for egne programmer \\
        \hline
        K09 & Slet kreditering & Kanaladmin og systemadmin skal kunne oprette/redigere/slette krediteringer under egen kanal \\
        \hline
        K10 & Søg efter personer & Alle skal kunne søge efter personer \\
        \hline
        K11 & Knyt personer til krediteringer & Personer skal kunne knyttes til krediteringer så man kan se hvilke programmer en person har deltaget i Systemadmin, kanaladmin og producer skal kunne se persondata som email og tlf. nr. \\
        \hline
        K12 & Link personer i den virkelige verden & Det skal være muligt at linke personer i krediteringer til personer i den virkelige verden, så der krediteres korrekt \\
        \hline
        K13 & Eksporter data & Brugere skal kunne eksportere data til forskellige formater såsom XML og CSV \\
        \hline
        K14 & Importering af data & Systemet skal kunne importere EPG data via TVTid.dk \\
        \hline
        K15 & Integration & Systemet skal kunne integreres med andre systemer (Yoursee Play, Boxer Play, osv.) \\
        \hline
        K16 & Notifikationer & Systemet skal sende notifikationer til relevante brugere \\
        \hline
        K17 & Sprogvalg & Systemet skal kunne understøtte flere sprog \\
        \hline
    \end{tabularx}
    \caption{Liste af krav fra overordnet kravspecifikation} 
    \label{table:kravliste}
\end{table} 

% ---------------------------- Overordnet brugsmønsterdiagram ----------------------
\subsection{Overordnet Brugsmønsterdiagram}

\begin{figure}[H]
    \centering
    \captionsetup{justification=centering}
    \includegraphics[scale=0.22]{figures/use-case.png}
    \caption[Overordnet brugsmønster over Creditoro systemet]%
    {Overordnet brugsmønster over Creditoro systemet 
    \par \small Menneskene er aktører. \\
    \small Cirklerne beskriver handlinger aktørne kan lave. \\
    \small Pilende betyder Extends, hvilket vil sige aktørerne arver funktionalitet}
    \label{fig:usecasemodel}
\end{figure}

% ---------------------------- Supplerende krav ------------------------------------
\subsection{Supplerende krav}

Her bruges FURPS for supplerende krav.
\begin{table}[ht]
    \begin{tabularx}{\textwidth}{|p{3cm}|X|}
        \hline
        \textbf{FURPS}           &    \textbf{Krav} \\
        \hline
        %What the customer wants! Note that this includes security-related needs.
        Functionality           & Skal kunne kreditere produktionsroller som er angivet af DRs Krediteringsregler \\
                                & Skal overholde GDPR \\
        \hline
        % How effective is the product from the standpoint of the person who must use it? Is it aesthetically acceptable? Is the documentation accurate and complete?
        Usability       & Systemet skal kunne understøtte flere sprog \\
        \hline
        % What is the maximum acceptable system downtime? Are failures predictable? Can we demonstrate the accuracy of results? How is the system recovered?
        Reliability     &  Hvis serveren til systemet genstarter, startes del-systemerne igen automatisk. Der vil ikke være behov for at ligge systemet ned regelmæssigt for at kunne foretage backup. \\
        \hline
        % How fast must it be? What's the maximum response time? What's the throughput? What's the memory consumption?
        Performance     &  Databasen skal kunne håndtere 10000 nye brugere - samt 15000 krediteringerer årligt i 25 år, uden at ofte brugte kald til REST Api'et bliver sløvt (reponsetid på mere end 300 ms) \\
        \hline
        % Is it testable, extensible, serviceable, installable, and configurable? Can it be monitored?
        Supportability  &  Systemet vil indeholde unit tests, og komme med en rapport over hvor stor en procendel der er dækket af dette. % Hvis der er mere tid kan vi indrage integrations tests?
        %Centraliseret logning (Elastic search, eller blot centraliseret system log?)
        Systemet vil blive forbundet til det centraliserede fejllognings system \texttt{Sentry}.
        Systemet er installerbart vha. Docker via Docker-compose. % (så vi blot kan sige docker-compose up -d for at få systemet op).
        Det vil være muligt at konfigurere system indstillinger via en \texttt{.env} (miljø) fil.
        En opsætningsguide vil være at finde sammen med kildekoden. 
        \\ \hline
        % The + reminds us of a few additional needs that a customer could have:
        % fx. Design constraints, Implementation requirements, interface requirements or physical requirements.
    \end{tabularx}
    \caption{FURPS}
    \label{tab:furps}
\end{table} 

\myworries{Følgende skal overvejes:}

\begin{itemize}
    \item Usability: Responsive UI
    \item Perfomance: The system should be able to handle 5000 concurrent users within a minute.
\end{itemize}

\newpage

\section{Detaljerede Krav}
\noindent
Detaljerede krav er en uddybelse af foregående afsnit "Overordnede Krav". I dette afsnit vil der udarbejdes detaljerede brugsmønsterbeskrivelser udfra de allerede udarbejdede overordnede brugsmønsterbeskrivelser. Der vil I dette afsnit blive fokuseret på det mest essentielle brugsmønster for systemet.

\noindent
Afsnittet vil derudover indeholde en detaljeret beskrivelse af de supplerende ikke-funktionelle krav. Til dette bliver FURPS+ modellen benyttet.


% ---------------------------- Detaljerede brugsmønstre ----------------------------
\subsection{Detaljerede Brugsmønstre} {\label{section: detailed_usemodels}}
I tabel \ref{tab:create_credits} ses en detaljeret beskrivelse af det essentielle brugsmønster "Opret Kreditering".
Beskrivelsen er udarbejdet ud fra aktørlisterne og brugsmønster beskrivelserne i inceptionsdokumentet (se bilag ? side ?). Detaljerede brugsmønster beskrivelserne bruges til at danne et overblik over hver enkel brugsmønsters cyklus. Her nævnes brugsmønstrets primære og sekundære aktører, de præ- og postkonditioner der er knyttet til brugsmønsteret - hvad skal være opfyldt, for at brugsmønsteret sættes i gang, og hvad sker der når brugsmønsteret er gennemført. Herudover beskrives hoved- og alternative hændelsesforløb. Hovedhændelsesforløbet beskriver forløbet, hvis der ingen problemer opstår. Alternativ hændelsesforløb beskriver eventuelle alternative hændelser i specifikke trin/steps i hovedhændelsesforløbet. \\

\noindent
Brugsmønstrene nedenfor er blevet valgt på baggrund af MosCow analysen i inceptionsdokumentet \myworries{(se bilag ? side ?)} baseret på systemets brugsmønstre. Det mest essentielle brugsmønstre er valgt ud fra brugsmønstrene i kategorien "must have". Det er vurderet at brugsmønstret "Opret Kreditering" er det mest essentielle bnrugsmønstre, da det denne funktion hele systemet er bygget op omkring.\\


% % ----------------------- Læs krediteringer -----------------------------------
% \subsubsection{Læs Krediteringer}
% \begin{longtable}[h]{|p{16cm}|}
%     \hline
%     \textbf{Brugsmønster:}  Læs kreditering \\ 
%     \hline
% 	\textbf{ID:} UC10 \\ 
% 	\hline
% 	\textbf{Primære aktører:} Systemadministrator, kanaladministrator, producer, royalty bruger, gæst \\ \hline
% 	\textbf{Sekundære aktører:} \\ \hline
% 	\textbf{Kort beskrivelse:} Alle skal kunne se krediteringen for programmerne. \\ \hline
% 	\textbf{Prækonditioner (Pre conditions):} \\ \hline
% \textbf{Hovedhændelsesforløb (main flow):} \\
% 1. Brugsmønstret starter når en aktør vil se krediteringer for et program \\
% 2. Aktøren søger efter programmet \\
% 3. Aktøren trykker på det ønskede program \\
% 4. Systemet checker hvilken rolle aktøren har \\
% 5. Aktøren bliver omdirigeret til den passende visning af krediteringen \\ \hline
%     \textbf{Postkonditioner (post conditions):} \\
%     En kreditering er blevet vist \\ \hline

% 	\textbf{Alternative hændelsesforløb (alternative flow):} \\
% Step 2: Hvis programmet ikke findes, får vedkommende besked om at programmet ikke findes \\ 
% \hline
% \caption{Brugsmønster: Læs kreditering}
% \label{table:read_credits}
% \end{longtable}

% % ----------------------- Opret krediteringer -----------------------------------
 \subsubsection{Opret Krediteringer}
\begin{longtable}[h] {|p{16cm}|}
 \hline
       \textbf{Brugsmønster:} Opret kreditering \\
    \hline
	\textbf{ID:} UC07 \\ \hline
	\textbf{Primære aktører:} Systemadministrator, kanaladministrator, Producer \\ \hline
	\textbf{Sekundære aktører:} \\ \hline
 	\textbf{Kort beskrivelse:} Produceren opretter en kreditering. Heri angives alle der har bidraget til produceringen af TV-programmet, filmen el. lign. \\ \hline
	\textbf{Prækonditioner (Pre conditions):} \\
 Aktøren skal være \hyperref[table:login]{logget ind på} systemet \\ \hline
 \textbf{Hovedhændelsesforløb (main flow):} \\
	1. Brugsmønstret starter når en administrator eller producer vil oprette en kreditering \\
	2. Aktøren trykker på knappen ‘Opret Kreditering’ \\
	3. Systemet checker aktørens rolle \\
	4. Aktøren er forbundet til en kanal, og angiver programmets titel \\
	5. Systemet checker om der allerede findes et program med den angivne titel \\ 
	6. Aktøren krediterer alle der har medvirket i produktionen af programmet \\ 
	7. Aktøren sender den færdige kreditering videre til godkendelse \\ \hline
\textbf{Postkonditioner (post conditions):} \\ 
	En kreditering er blevet sendt videre til godkendelse \\ \hline
	\textbf{Alternative hændelsesforløb (alternative flow):} \\
tep *: Aktøren kan til enhver tid afbryde oprettelsen af krediteringen \\
Step 4: Hvis aktøren er systemadministrator, er vedkommende ikke forbundet til en kanal, og kan skifte hvilken kanal krediteringen skal oprettes ved. \\

Step 5: Hvis programmets titel allerede eksisterer, gøres aktøren opmærksom på dette. \\

Step 8: Hvis krediteringen afvises, laves de fornødne ændringer, og den nye kreditering sendes videre til godkendelse. \\
\hline
\caption{Brugsmønster: Opret kreditering}
\label{tab:create_credits}
\end{longtable}

% % ----------------------- Opret Producer -----------------------------------
% % navneord: l72: producer, l74: systemadministrator, l76: kanaladministrator, l76: krediteringer, l76: systemet, l78: Aktøren, l79: flow, l80: brusmønster, l81: administrator
% % 
% \subsubsection{Opret Producer} % hi kechr19
% \begin{longtable}[h]{|p{16cm}|} \hline
% \textbf{Brugsmønster:}  Opret producer \\ \hline
% \textbf{ID:} UC03 \\ \hline
% \textbf{Primære aktører:} Systemadministrator, kanaladministrator \\ \hline
% \textbf{Sekundære aktører:} \\ \hline
% \textbf{Kort beskrivelse:} System- eller kanaladministrator opretter en producer, som kan oprette       krediteringer i systemet \\ \hline
% \textbf{Prækonditioner (Pre conditions):} \\
% Aktøren skal være logget på systemet \\ \hline
% \textbf{Hovedhændelsesforløb (main flow):} \\
%     1. Dette brugsmønster starter, når en administrator vil oprette en ny producer \\
%     2. Administrator indtaster producerens oplysninger \\ 
%     3. Administratoren opretter produceren \\ \hline
% \textbf{Postkonditioner (post conditions):} \\
%     En producer er oprettet i systemet \\ \hline
% \textbf{Alternative hændelsesforløb (alternative flow):} \\ Step 3. Produceren eksisterer allerede,     administrator får besked om dette ved oprettelse. Oprettelsen bliver ikke gennemført \\
% \hline
% \caption{Brugsmønster: Opret Producer}
% \label{table:create_producer}
% \end{longtable}


% % ---------------------- Se Personinformationer ---------------------------------
% \subsubsection{Se Personinformationer}
% \begin{longtable}[h]{|p{16cm}|}
%     \hline
%     \textbf{Brugsmønster:}  Se Personinformationer \\ 
%     \hline
% 	\textbf{ID:} UC04 \\ 
% 	\hline
% 	\textbf{Primære aktører:} Systemadministrator, Kanaladministrator, Royality Bruger \\ \hline
% 	\textbf{Sekundære aktører:} \\ \hline
% 	\textbf{Kort beskrivelse:} System-, kanaladministrator og Royality Bruger får vist personinformation om en bestemt person. \\ \hline
% 	\textbf{Prækonditioner (Pre conditions):} \\ \hline
% \textbf{Hovedhændelsesforløb (main flow):} \\
% 1. Dette brugsmønster starter, når en primær aktør vil se information om en bestemt person \\ 
% 2. Den primære aktør søger efter en bestemt person ved at indtaste navn \\ 
% 3. Personer med det pågældende navn bliver vist \\ 
% 4. Den primære aktør vælger den person, der eftersøges \\
% 5. Informationen om den gældene person bliver vist \\ \hline
%     \textbf{Postkonditioner (post conditions):} \\
%     Personinformationer er blevet vist \\ \hline

% 	\textbf{Alternative hændelsesforløb (alternative flow):} \\
% step 4. Ingen person med det eftersøgte navn findes. Så bliver der angivet at der ikke findes nogle personer\\ \hline
% \caption{Brugsmønster: Se Personinformationer}
% \label{table:read_personinfo}
% \end{longtable}



% %------------------------ Godkend nye krediteringer -------------------------
% \subsubsection{Godkend nye krediteringer}
% \begin{longtable}[h]{|p{16cm}|}
%     \hline
%     \textbf{Brugsmønster:}  Godkend nye krediteringer \\ 
%     \hline
% 	\textbf{ID:} UC05 \\ 
% 	\hline
% 	\textbf{Primære aktører:} Systemadministrator, kanaladministrator \\ \hline
% 	\textbf{Sekundære aktører:} \\ \hline
% 	\textbf{Kort beskrivelse:} Efter Produceren har sendt en kreditering ind til godkendelse, kan kanaladministratoren enten godkende eller afvise krediteringen. Hvis krediteringen godkendes, gøres krediteringen offentlig. Hvis krediteringen afvises, skal kanaladministratoren skrive en meddelelse om hvad der skal ændres. \\ \hline
% 	\textbf{Prækonditioner (Pre conditions):} \\
% 	1. Kanaladministratoren skal være logget ind \\ 
% 	2. Produceren skal have sendt en kreditering ind til godkendelse \\
% 	\hline
% \textbf{Hovedhændelsesforløb (main flow):} \\
% 1. Dette brugsmønster starter når en producer har sendt en kreditering ind til godkendelse \\ 
% 2. Kanaladministratoren gennemgår krediteringen \\ 
% 3. Hvis alt er i orden, trykker kanaladministratoren på godkend.\\ \hline
% \textbf{Postkonditioner (post conditions):} \\
%     En kreditering er blevet godkendt og offentliggjort \\ \hline
% \textbf{Alternative hændelsesforløb (alternative flow):} \\
% Step 3: Hvis noget skal ændres i krediteringen, trykker kanaladministratoren på afvis. Derefter kan kanaladministratoren skrive en meddelelse om rettelser \\
% \hline
% \caption{Brugsmønster: Godkend nye krediteringer}
% \label{table:approve_credit}
% \end{longtable}


% %------------------------ Opret Person -------------------------
% \subsubsection{Opret Person}
% \begin{longtable}[h]{|p{16cm}|}
%     \hline
%     \textbf{Brugsmønster:} Opret Person \\ 
%     \hline
% \textbf{ID:} \myworries{UC06} \\ 
% 	\hline
% 	\textbf{Primære aktører:} Systemadministrator, kanaladministrator, Producer \\ \hline
%     \textbf{Sekundære aktører:} \\ \hline
%     \textbf{Kort beskrivelse:} Når en producer opretter krediteringer og en person der ikke allerede findes i systemet skal krediteres, skal produceren oprette vedkommende i systemet. \\ \hline
% 	\textbf{Prækonditioner (Pre conditions):} \\
% 	1. System-, kanaladministrator eller producer skal være \hyperref[table:login]{logget ind} \\
% 	\hline
% \textbf{Hovedhændelsesforløb (main flow):} \\
% 1. Dette brugsmønster starter når en system-, kanaladministrator eller producer skal indskrive krediteringer \\
% 2. Systemet checker om vedkommende allerede findes i systemet \\
% 3. Aktøren trykker 'Opret Person' \\
% 4. Opret Person-vinduet popper op \\
% 5. Aktøren udfylder de nødvendige informationer (navn, beskæftigelse, email, tlf.nr., osv.) \\
% 6. Aktøren trykker 'Færdig' \\
% 7. Personen bliver automatisk indskrevet det pågældende sted i krediteringen \\
% \hline
% \textbf{Postkonditioner (post conditions):} \\
%     En person er blevet oprettet i systemet \\ \hline
% \textbf{Alternative hændelsesforløb (alternative flow):} \\
% Step 5: Hvis aktøren blot har lavet en stavefejl, og en person med samme email, tlf.nr. osv. Findes, bliver aktøren oplyst om dette og oprettelse af personen afbrydes. \\
% Step 1-6: Aktøren kan til enhver tid afbryde oprettelsen \\
% \hline
% \caption{Brugsmønster: Opret Person}
% \label{table:create_person}
% \end{longtable}


% %------------------------ Login -------------------------
% \subsubsection{Login}
% \begin{longtable}[h]{|p{16cm}|}
%     \hline
%     \textbf{Brugsmønster:} Login \\ 
%     \hline
% \textbf{ID:} \myworries{UC07} \\ 
% 	\hline
% 	\textbf{Primære aktører:} Gæst \\ \hline
%     \textbf{Sekundære aktører:} \\ \hline
%     \textbf{Kort beskrivelse:} En gæst kan logge ind hvis vedkommende har en bruger i systemet. \\ \hline
% 	\textbf{Prækonditioner (Pre conditions):} \\
% 	Vedkommende skal ikke være logget ind \\
% 	\hline
% \textbf{Hovedhændelsesforløb (main flow):} \\
% 1. Dette brugsmønster starter når en aktør vil logge ind \\
% 2. Aktøren trykker login \\
% 3. Aktøren bliver omdirigeret til login-siden \\
% 4. Aktøren udfylder login-oplysninger \\
% 5. Aktøren trykker ’Login’ \\
% 6. Aktøren bliver omdirigeret til startsiden \\
% \hline
% \textbf{Postkonditioner (post conditions):} \\
%     Aktøren er logget ind \\ \hline
% \textbf{Alternative hændelsesforløb (alternative flow):} \\
% Step 6: Hvis login-oplysningerne er forkerte, forbliver aktøren på login-siden og får en meddelelse om at brugernavn eller password var forkert \\
% Step *: Aktøren kan til enhver tid afbryde login \\
% \hline
% \caption{Brugsmønster: Login}
% \label{table:login}
% \end{longtable}


% %------------------------ Logout -------------------------
% \subsubsection{Logout}
% \begin{longtable}[h]{|p{16cm}|}
%     \hline
%     \textbf{Brugsmønster:} Logout \\ 
%     \hline
% \textbf{ID:} \myworries{UC08} \\ 
% 	\hline
% 	\textbf{Primære aktører:}  Systemadministrator, kanaladministrator, producer, Royalty Bruger \\ \hline
%     \textbf{Sekundære aktører:} \\ \hline
%     \textbf{Kort beskrivelse:} En aktør kan logge ud af desktopklienten \\ \hline
% 	\textbf{Prækonditioner (Pre conditions):} \\
% 	Vedkommende skal være \hyperref[table:login]{logget ind} \\ 
% 	\hline
% \textbf{Hovedhændelsesforløb (main flow):} \\
% 1. Dette brugsmønster starter når en aktør vil logge ud \\
% 2. Aktøren trykker logout \\
% 3. Aktøren bliver omdirigeret til startsiden \\
% \hline
% \textbf{Postkonditioner (post conditions):} \\
%     Aktøren er logget ud \\ \hline
% \textbf{Alternative hændelsesforløb (alternative flow):} \\
% \hline
% \caption{Brugsmønster: Logout}
% \label{table:logout}
% \end{longtable}


% ---------------------------- Detaljeret beskrivelse af supplerende krav ------------------------------------

\subsection{Detaljeret beskrivelse af supplerende krav}

\noindent I tabel \ref{tab:furps+} ses de detaljerede beskrivelser af supplerende ikke-funktionelle krav i form af en FURPS+ model. modellen er med til at give et overblik over hvilke supplerende ikke-funktionelle krav systemet skal overholde, for at være tilstrækkeligt i forhold til TV2 og gruppens forventninger.
FURPS+ er en udviddet udgave af FURPS modellen. Udover kravene i FURPS modellen i afsnit \ref{supplerende_krav}, er der i tilføjet 4 nye kategorier - Design constraints, Implementation requirements, Interface requirements og Physical requirements. Derudover har hver af de ikke-funktionelle krav fået tildelt et nummer, for lettere reference. \\
Af de tilføjede kategorier er nye krav udformet. Systemet skal benytte en relationel databse. Databasen skal skrives i programmeringssproget SQL og laves i PostgreSQL. Desktop clienten bygges i programmeringssproget Java og REST Apt'et bygges i Python.\\
Systemet vil blive unit testet, for at sikre en høj kvalitet af implementeret kode. Systemet vil blive forbindet til det centraliserede fejlognings system Sentry.\\
Systemet skal kunne trække EPG data fra TVTid.dk via en EPG Poller og sende dataen til REST Api'et.

\begin{center}
    \begin{longtable}[h]{|p{4cm}|p{1cm}|p{11cm}|}
        \hline
        \textbf{FURPS+}                     & \textbf{\#}   & \textbf{Krav} \\ 
        \hline
        \multirow{2}{0}{Functionality}      & S01           & Skal kunne kreditere produktionsroller som er angivet af DRs Krediteringsregler  \\ \cline{2-3} 
                                            & S02           & Skal overholde GDPR \\ \hline
        \multirow{2}{0}{Usability}          & S03           & Systemet skal kunne understøtte flere sprog \\ \cline{2-3}
                                            & S04           & Systemet skal have en responsiv brugergrænseflade (UI) \\ \hline
        Reliability                         & S05           & Hvis serveren til systemet genstarter, startes del-systemerne igen automatisk. Der vil ikke være behov for at ligge systemet ned regelmæssigt for at kunne foretage backup. \\ \hline
        \multirow{2}{0}{Performance}       & S06           & Databasen skal kunne håndtere 10000 nye brugere - samt 15000 krediteringerer årligt i 25 år, uden at ofte brugte kald til REST Api'et bliver sløvt (reponsetid på mere end 300 ms) \\ \cline{2-3}
                                            & S07           & Systemet skal kunne håndtere 5000 brugere indenfor et minut\\ \hline
        \multirow{3}{0}{Supportability}     & S08           & Systemet er installerbart vha. Docker via Docker-compose\\ \cline{2-3}
                                            & S09           & Det vil være muligt at konfigurere system indstillinger via en \texttt{.env} (miljø) fil. En opsætningsguide vil være at finde sammen med kildekoden. \\ \hline
        Design constraints                  & S10           & Systemet skal benytte en relationel database. Databasen skal laves i PostgreSQL og skrives i programmeringssproget SQL\\ \hline
        \multirow{3}{0}{Implementation requirements}   & S08   & Systemet vil indeholde unit tests, og komme med en rapport over hvor stor en procentdel der er dækket af dette.\\ \cline{2-3}
                                            & S11           & Systemet vil blive forbundet til det centraliserede fejllognings system \texttt{Sentry}.\\ \cline{2-3}   
                                            & S12           & Desktop klienten implementeres i programeringssproget Java. REST API implementeres i Python\\ \hline
        Interface requirements              & S13           & Systemet skal trække EPG data fra TVTid.dk via en EPG Poller. \\ \hline
        Physical requirements               & S14           & Ingen\\ \hline
    \caption{FURPS+}
    \label{tab:furps+}
    \end{longtable}
\end{center}
\newpage

\section{Analyse}
I dette afsnit bliver overvejelser, beslutninger og resultater vedrørende den statiske side af analysemodel og den dynamiske side af analysemodel berørt. 

\subsection{Brugsmønsteranalyse}
Der vil i dette afsnit blive lavet en brugsmønsteranalyse på de detaljerede brugsmønstre fra afsnit \ref{section: detailed_usemodels}, der har til formål at finde klasse, struktur og adfærd i et system. Det første der gøres er at finde klasser:

\subsubsection{Klassekandidater}
Når man skal finde potentielle klasser, kan der tages brug af en navneordsanalyse, hvor navneord i de detaljerede brugsmønster er potentielle klasserkandidater. Listen af potentielle klasser fra navneordsanalysen af brugsmønstrene ses i tabel \ref{table:class_candidates}.

\begin{table}[ht]
    \begin{tabularx}{\textwidth}{|>{\RaggedRight}p{4cm}|>{\RaggedRight}p{4cm}|>{\RaggedRight}X|}
        \hline
        \textbf{Klassekandidat} & \textbf{Attributter} & \textbf{Kommentar} \\
        \hline
        % Læs kreditering -----------------------------------------------------------
        \multirow{2}{*}{Bruger}      & \texttt{uuid}, \texttt{e-mail}, \texttt{password}, \texttt{rolle} & Indebærer alle aktørerne (Systemadmin, kanaladmin, producer og royalty bruger).\\
        \hline
        \multirow{2}{*}{Produktion}  & \texttt{uuid}, \texttt{titel}, \texttt{kanal\_id} & Et program indebærer alt hvad der bliver vist på TV. (Film, serier osv) \\
        \hline
        \multirow{3}{*}{Kreditering} & \texttt{person\_id}, \texttt{job}, \texttt{produktion\_id} & En kreditering er et job udført af en person til en produktion (hvor job  kan være lydmand) \\
        \hline
        % Opret Kreditering ---------------------------------------------------------
        \multirow{2}{*}{Kanal} & \multirow{2}{*}{\texttt{ID}, \texttt{navn}} & Angiver hvilken TV-kanal et program bliver afspillet på \\
        \hline
        % Opret Person --------------------------------------------------------------
        \multirow{3}{*}{Person} & Personinformation (navn, beskæftigelse, e-mail, tlf.nr. osv)  & De der medvirker til produceringen af en produktion (kameramand, lydmand, etc.)\\
        \hline
        % Log ud --------------------------------------------------------------------
            % Nur duplications
        %Log in ---------------------------------------------------------------------
            % Nur Duplications
        % Opret Producer ------------------------------------------------------------
            % Nur Duplications
        % Se personinformation ------------------------------------------------------
            % Nur Duplications
        % Godkend nye krediteringer -------------------------------------------------
    \end{tabularx}
    \caption{Klasssekandidater}
    \label{table:class_candidates}
\end{table}

% vi behøver ikke at bruge  mere :)

Her er der tildelt nogle attributter, blandt andet fra kasserede klassekandidater, som klasserne bør indeholde.

\subsubsection{Kasserede klassekandidater}
De resterende potentielle klasser der blev fundet under navneordsanalysen er enten synonymer, operationer eller egner sig bedre som attributter til de valgte potentielle klasser. Disse kan ses i tabel \ref{table:deleted_class_candidates}

\begin{table}[H]
    \begin{tabularx}{\textwidth}{|>{\RaggedRight}X|>{\RaggedRight}X|}
        \hline
        \textbf{Kasserede klassekandidater} & \textbf{Kommentar} \\
        \hline
        Gæst & Påvirker ikke systemet - Kan kun se krediteringerne\\
        \hline
        Systemet & Systemet  \\
        \hline
        Rolle   & Attribut til Bruger-klassen\\
        \hline
        Visning  & En Operation\\
        \hline
        Film & Synonym til Produktion-klassen\\ 
        \hline
        Program & Synonym til Produktion-klassen\\ 
        \hline
        Knap & En del af Desktop-klienten\\
        \hline
        Programtitel & Attribut til Program-klassen\\
        \hline
        Vedkommende & Synonym for Bruger-klassen\\
        \hline
        Aktøren &  Synonym for Bruger-klassen\\
        \hline
        Person-vinduet & En del af Desktop-klienten\\
        \hline
        Informationer (navn, beskæftigelse, email, tlf.nr., osv.) &  Attributter til Person-klassen\\
        \hline
        Oprettelse & En operation \\
        \hline
        Desktopklienten &  Systemet\\
        \hline
        Startsiden &  En del af Desktop-klienten\\
        \hline
        Login-side & En del af Desktop-klienten\\
        \hline
        Login-oplysninger & Attribut til Bruger-klassen\\
        \hline
        Brugernavn & Attribut til Bruger-klassen\\
        \hline
        Password & Attribut til Bruger-klassen\\
        \hline
        Flow & Beskriver et hændelsesforløb\\
        \hline
        Navn & Attribut \\
        \hline
        Primær aktør &  Synonym for Bruger-klassen\\
        \hline
        Personinformation &  Attributter til Person-klassen\\
        \hline
        Godkendelse & En operation \\
        \hline
        Meddelelse & En operation \\
        \hline
        Systemadministrator & En rolle i systemet - bliver håndteret af Bruger-klassen \\
        \hline
        Kanaladministrator & En rolle i systemet - bliver håndteret af Bruger-klassen \\
        \hline 
        Producer & En rolle i systemet - bliver håndteret af Bruger-klassen \\
        \hline
        Royalty Bruger & En rolle i systemet - bliver håndteret af Bruger-klassen \\
        \hline
    \end{tabularx}
    \caption{Kasserede klassekandidater}
    \label{table:deleted_class_candidates}
\end{table}


Her er det besluttet, at kandidaterne 'Brugernavn' og 'Password' begge er attributter til brugerklassen, og kandidaten 'Gæst' ikke skal være en klasse for sig, da denne klassekandidat kun kan se krediteringerne. Gæster skal ikke have specielle rettigheder eller login-oplysninger, som brugernavn og password.


\subsubsection{Klassekategori koncept}
De potentielle klasser der arbejdes videre med kan opdeles i kategorier. Dette er med til at skabe et overblik over klassernes ‘placering’ i systemet samt skabe associationer mellem koncepter og potentielle klasser.
Kategorilisten kan ses i tabel \ref{table:class_categories}.

\begin{table}[H]
    \begin{tabularx}{\textwidth}{|>{\RaggedRight}X|>{\RaggedRight}X|}
        \hline
        \textbf{Klassekategori koncept} & \textbf{Eksempel} \\
        \hline
        Forretnings overførsel & Ingen\\
        \hline
        Overførselslinje ting & Ingen\\
        \hline
        Produkt & Ingen \\
        \hline
        Hvor bliver overførslen optaget  & Ingen \\
        \hline
        Folkets roller & System- \& kanaladministrator, Producer, Royalty Bruger\\
        \hline
        Sted for overførsel & Ingen \\
        \hline
        Bemærkelsesværdige begivenheder & Ingen \\
        \hline
        Fysiske objekter & Ingen \\
        \hline
        Beskrivelse af ting & Krediteringer\\
        \hline
        Kataloger & Database med krediteringer\\
        \hline
        Lager af ting & Databasen\\
        \hline
        Ting på lageret & Krediteringer\\
        \hline
        Andre samarbejdssystemer & Ingen\\
        \hline
        Optagelser af finans, arbejde, kontrakter og juridiske sager & Ingen \\
        \hline
        Finansielle instrumenter & Ingen\\
        \hline
        Arbejdsplaner, manualer, dokumenter der er regulært refereret til for at performere arbejde & Ingen\\
        \hline
    \end{tabularx}
    \caption{Klassekategorier}
    \label{table:class_categories}
\end{table}

\subsubsection{Beskrivelse af klasser}
I tabel \ref{tab:user_class_description} er der givet et eksempel på, hvordan hver klasse kan komme til at se ud. De resterende kan findes i bilagene.
 
% ------------------------- Bruger --------------------------------
\begin{table}[H]
    \begin{tabularx}{\textwidth}{|>{\RaggedRight}p{3cm}|>{\RaggedRight}X|}
        \hline
        \textbf{Navn} & Bruger\\
        \hline
        \textbf{Definition} &  En bruger har et UUID, en email-adresse, et password og en rolle. \\
        \hline
        \textbf{Eksempel} & En bruger kan se ud som følgende: \\
                          & \texttt{UUID}: fbf50f43-9f1c-41e1-abce-cca32b836ef0 \\
                          & \texttt{Email}: someperson@somemail.com\\
                          & \texttt{Password}: Passw0rd\\
                          & \texttt{Rolle}: Kanaladministrator \\
        \hline
        \textbf{Andet} & Nej\\
        \hline
    \end{tabularx}
    \caption{Beskrivelse af Bruger-klassen}
    \label{tab:user_class_description}
\end{table}

\subsubsection{Analysemodel}


Ud fra navneordsanalysen er de potentielle klasser og deres attributter fundet. Dette vises i analysemodellen, som kan ses på figur \ref{fig:analysemodel} , med tilhørende relationer mellem klasserne. Modellen har til formål at danne et overblik over klasserne, deres attributter og deres relationer i systemet, samt at fungere som en model der kan bruges til videreudvikling af systemet. \\

\begin{figure}[h]
\centering
\includegraphics[scale=0.4]{figures/analysemodel.png}
\caption{Analysemodel}
\label{fig:analysemodel}
\end{figure}
\newpage

\subsection{Brugsmønsterrealisering}

% -----------------------------------------------------------------------------------------------------------------
\subsubsection{Systemsekvensdiagram}
Et systemsekvensdiagram er et sekvensdiagram der viser systemhændelserne for ét scenarie af et brugsmønster. Diagrammet viser hvordan aktørerne interagerer med systemet for at opfylde brugsmønstret. Diagrammet viser systemet som en ‘black box’, hvilket betyder at man ikke kan se
hvad der sker inde i systemet, man kun hvad der sker udenfor systemet. På diagrammet ses det, hvordan aktørerne genererer systembegivenheder og hvad systemets output er. Ydermere viser diagrammet den ‘tidslinje’ begivenhederne sker i. \\
\myworries{Et eksempel på et systemsekvensdiagram kan ses nedenfor.} \\


\begin{figure}[H]
\centering
\includegraphics[scale=0.4]{figures/systemsekvensdiagrammer/opretKreditering.PNG}
\caption{Systemsekvensdiagram for "Opret Kreditering"}
\label{fig:systemsekvensdiagram_opretKreditering}
\end{figure}


\myworries{Her kan vi se, at aktøren (Systemadministrator, kanaladministrator eller Producer) kalder metoden opretKreditering ind til systemet, som returnerer at en kreditering er blevet oprettet. Vi kan altså ikke se hvordan krediteringen bliver oprettet, og vi holder os dermed til hvad der sker uden for systemet. Ud over det ovenstående (system)sekvensdiagram har vi valgt at medtage systemsekvensdiagrammer for \textit{godkendKreditering}, \textit{opretProducer}, \textit{sePersonInfo} og \textit{læsKreditering}. } \\

\myworries{Efter systemsekvensdiagrammerne er lavet, finder man de operationer systemsekvensdiagrammet kræver. Vi kigger altså her ikke længere blot udenfor systemet, men ser på præcis hvordan 'flowet' for brugsmønstret udvikles. Til dette skal vi kigge på kontrakter for systemfunktioner.} \\

\myworries{ Indsæt tekst om hvilke systemsekvensdiagrammer vi har valgt at medbringe her i teksten, hvorfor og at de andre kan findes i bilag :-) - Sammenkobel den her tekst med den tekst der er i brugsmønsterrealisering - kechr  } \\

\begin{figure}[H]
\centering
\includegraphics[scale=0.43]{figures/systemsekvensdiagrammer/læsKreditering.PNG}
\caption{Systemsekvensdiagram for "Læs kreditering"}
\label{fig:read_credit}
\end{figure}


% -----------------------------------------------------------------------------------------------------------------
\subsubsection{Kontrakter for systemfunktioner}
En systemoperationskontrakt beskriver en operations ansvar, altså hvad en operation har forpligtet sig til. Kontrakten lægger vægt på hvad en operation ændrer på, og ikke på hvordan det ændre sig. En kontrakt kan derfor anses som værende en formel beskrivelse af en operation.\\
Systemoperationskontrakten indeholder navnet på operationen, krydsreferencer til de relevante brugsmønstre, beskrivelse af ansvaret, det output operationen genererer, samt pre- og postkonditioner for operationen. \\
Pre- og postkondinitionerne er stilbilleder af systemet på det givne tidspunkt operationen bliver kaldt. De beskriver altså systemets tilstand før og efter at operationen har kørt. Postkonditioner skal altid noteres i datid, som f.eks. “Købet blev foretaget”, da det er en ting der er sket, og ikke sker.\\

\noindent
\myworries{Det er valgt at lave operationskontrakter for de samme operationer som ved systemsekvensdiagrammerne, da disse operationer viser essensen af den logik der skal implementeres.} \\

\myworries{Skriv dette om - Sammenkoble med den tekst der står under systemsekvensdiagrammer -- Eventuelt skriv det under 10. brugsmønsterrealisering, at vi har valgt at arbejde med de brugsmøsnter/operationer/we vi har - begrund uddybende - kechr}

% Skal "overskriften" ikke bare fjernes fra nedenstående tabeller? - kechr
% Skal vi bare indsætte en enkelt operationskontrakt, og smide resten i bilagene?

%------------------------ Læs krediteringer -------------------------------
\begin{table}[H]
    \begin{tabularx}{\textwidth}{|p{4cm}|X|}
        \hline
        \multicolumn{2}{|X|}{\textbf{Læs Kreditering}}\\ 
        \hline
        \textbf{System operation}       & \textbf{læsKreditering} \\ \hline
        \textbf{Krydshenvisning}        & Use case: Læs kreditering \\ \hline
        \textbf{Ansvar}                 & At vise eksisterende krediteringer, hvis følgende betingelser er sande \\ 
                                        & \\
                                        & \quad 1. Den søgte kreditering findes i systemet\\
                                        & \\
                                        & Hvis ikke overstående er sande, skal det sendes en besked til brugeren at den søgte kreditering ikke findes i systemet\\\hline
        \textbf{Output}                 & 1. Krediteringen vises\\ 
                                        & Alternativt: Besked sendt ud til brugeren\\ \hline
        \textbf{Prækonditioner}         & Ingen \\ \hline
        \textbf{Postkonditioner}        & En kreditering vil blive vist \\ \hline
    \end{tabularx}
    \caption{Systemfunktionskontrakt 'læsKreditering'}
    \label{tab:kontrakter_læs_kreditering}
\end{table}


%------------------------ Opret kreditering -------------------------------
\begin{table}[H]
    \begin{tabularx}{\textwidth}{|p{4cm}|X|}
        \hline
        \multicolumn{2}{|X|}{\textbf{Opret kreditering}}\\
        \hline
        \textbf{System operation}       & \textbf{opretKreditering} \\ \hline
        \textbf{Krydshenvisning}        & Use case: Opret Kreditering \\ \hline
        \textbf{Ansvar}                 & At oprette krediteringer og sende den videre til godkendelse hvis prækonditionen er opfyldt, og                                       følgende betingelser er sande: \\
                                        & \\
                                        & \quad 1. Minimumskravene er opfyldt\\
                                        & \quad 2. Alle oplysninger er indtastet korrekt \\
                                        & \\
                                        & Hvis ikke ovenstående er sande, oprettes krediteringen ikke, og brugeren bliver informeret herom \\ \hline
        \textbf{Output}                 & \quad 1. Produceren får besked om krediteringen er oprettet \\ 
                                        & \quad 2. System- og/eller kanaladministrator får besked om en nyoprettet kreditering \\\hline
        \textbf{Prækonditioner}         & Logget ind som kanal- eller systemadministrator \\ \hline
        \textbf{Postkonditioner}        & En ny kreditering er oprettet i systemet \\ \hline
    \end{tabularx}
    \caption{Systemfunktionskontrakt 'Opret kreditering'}
    \label{tab:kontrakter_opret_kreditering}
\end{table}


%------------------------ Opret producer -------------------------------
\begin{table}[H]
    \begin{tabularx}{\textwidth}{|p{4cm}|X|}
        \hline
        \multicolumn{2}{|X|}{\textbf{Opret producer}}\\
        \hline
        \textbf{System operation}       & \textbf{opretProducer} \\ \hline
        \textbf{Krydshenvisning}        & Use case: Opret producer \\ \hline\textbf{}
        \textbf{Ansvar}                 & At oprette en producer, hvis prækonditionen er opfyldt og følgende betingelser er sande: \\ 
                                        & \\ 
                                        & \quad 1. Produceren eksisterer ikke i systemet i forvejen \\
                                        & \quad 2. Alle oplysninger er indtastet korrekt \\
                                        & \\
                                        & Derefter give brugeren besked om det lykkedes at oprette en producer eller ej \\\hline
        \textbf{Output}                 & Besked om der er blevet oprettet en producer eller ej \\ \hline
        \textbf{Prækonditioner}         & Vedkommende der prøver at oprette en producer skal være logget ind som 'Kanal- eller                                                  Systemadministrator' \\ \hline
        \textbf{Postkonditioner}        & En ny producer er oprettet i systemet \\ \hline
    \end{tabularx}
    \caption{Systemfunktionskontrakt 'Opret producer'}
    \label{tab:kontrakter_opret_producer}
\end{table}


%------------------------ Se Personinformationer -------------------------
\begin{table}[H]
    \begin{tabularx}{\textwidth}{|p{4cm}|X|}
        \hline
        \multicolumn{2}{|X|}{\textbf{Se Personinformationer}}\\
        \hline
        \textbf{System operation}       & \textbf{sePersonInfo} \\ \hline
        \textbf{Krydshenvisning}        & Use case: Se personinformation \\ \hline
        \textbf{Ansvar}                 & At vise respektive persondata om den søgte person, hvis prækonditionen og betingelsen er opfyldt:\\
                                        & \\
                                        & \quad 1. Personen er logget ind som enten producer, kanal- eller systemadministrator\\
                                        & \\
                                        & Hvis ikke overstående er opfyldt, vises ikke personfølsomme informationer, som navn, roller i film/serier som personen har medvirket i\\ \hline
        \textbf{Output}                 & Viser en information om en person \\ \hline
        \textbf{Prækonditioner}         & Person er fundet i systemet \\ \hline
        \textbf{Postkonditioner}        & Alt efter ens rolle i systemet vil forskelligt data blive vist. Begrænset hvis man er besøgende, alt hvis man er producer, royalty bruger, kanal- og systemadministrator \\ \hline
    \end{tabularx}
    \caption{Systemfunktionskontrakt 'se personinformationer'}
    \label{tab:kontrakter_se_personinformationer}
\end{table}

%------------------------ Godkend nye krediteringer -------------------------
\begin{table}[H]
    \begin{tabularx}{\textwidth}{|p{4cm}|X|}
        \hline
        \multicolumn{2}{|x|}{\textbf{Godkend ny kreditering}}\\
        \hline
        \textbf{System operation}       & \textbf{godkendKreditering} \\ \hline
        \textbf{Krydshenvisning}        & Use case: Godkend Kredtiteringer \\ \hline
        \textbf{Ansvar}                 & At godkende eller afvise nye krediteringer hvis prædonditionerne er sande og betingelsen opfyldt:  \\ 
                                        & \\
                                        & \quad 1. En kreditering er oprettet af en producer \\
                                        & \\
                                        & Hvis ikke ovenstående er opfyldt, vil der ikke blive vist nye krediteringer til godkendelse. \\ \hline
        \textbf{Output}                 & Besked om krediteringen er godkendt \\ \hline
        \textbf{Prækonditioner}         & 1. Er logget ind som system- eller kanaladministrator \\ \hline
        \textbf{Postkonditioner}        & Krediteringen er enten godkendt eller afvist og den ansvarlige producer er bliver informeret herom \\ \hline
    \end{tabularx}
    \caption{Systemfunktionskontrakt 'Godkend nye krediteringer'}
    \label{tab:kontrakter_Godkend_nye_krediteringer}
\end{table}



% -----------------------------------------------------------------------------------------------------------------
\subsubsection{Operationssekvensdiagram}
Operationssekvensdiagrammet viser systemet som en ”white box”, hvor man kan se hvad der sker inde i systemet. Sekvensdiagrammet bruges til at identificere systemfunktioner, da de begivenheder der vises i diagrammet er de funktioner systemet skal indeholde. Et eksempel på et operationssekvensdiagram for "Læs Kreditering" kan ses nedenfor.

\begin{figure}[ht] % <--- HUSK AT ÆNDRE MIG :-)
\centering
\includegraphics[scale=1]{figures/Operationssekvensdiagrammer/læsKreditering.pdf}
\caption{Operationssekvensdiagram: Læs kreditering}
\label{fig:op_read_credit}
\end{figure}

% -----------------------------------------------------------------------------------------------------------------
\subsubsection{Revurderet analysemodel}

\myworries{De fundne metoder tilføjer vi til analysemodellen vi lavede tidligere.} \\
% ---------------------------------------------------------------INDSÆT MODEL
\myworries{Da der ikke er fundet nogle nye klasser under brugsmønsterrealiseringen, opdateres modellen ikke yderligere. } \\
\newpages
\newpage

\section{Design}
Denne sektion skal indeholde:

\begin{itemize}
    \item Overvejelser, beslutninger og resultater vedr. Softwarearkitektur, Subsystemdesign, design af persistens. 
\end{itemize}{}
\newpage

\section{Databasedesign}
Denne sektion skal indeholde:

\begin{itemize}
    \item Overvejelser, beslutninger og resultater vedr. tabeldesign og SQL-forespørgsler 
\end{itemize}{}
\newpage

\section{Implementering}
Denne sektion skal indeholde:

\begin{itemize}
    \item Omfatter afsnittet overvejelser, beslutninger og resultater vedr.  konvertering fra design til kode illustreret gennem udvalgte centrale eksempler, samt andre vigtige implementeringsbeslutninger. 
    \item Implementering af database.
\end{itemize}{}
\newpage

\section{Test}
\subsection{1. Iteration af desktop-klient testning}
\subsubsection{Test af Views enum}
En af de test der bliver udført i core pakken er ViewsTest. Testen har ikke brug for nogen forberedelse da den kun skal checke om de værdier der er i Views (et enum) er en af dens værdier. Det gør den ved først at iterere igennem alle Views' værdier og for hver værdi kalder metoden isView.\\

\begin{code}[caption=ViewsTest.jar, firstnumber=10]
@Test
public void testEnumValues() {
    for (Views view : Views.values()) {
        assertTrue(isView(view));
    }
}
\end{code}

Denne metoder tager et view og tjekker alle Views' værdier og tjekker hvis værdien er lig den indsatte værdi. Hvis den findes returnere den true eller false.

\begin{code}[caption=ViewsTest.jar, firstnumber=17]
public boolean isView(Views view) {
    for (Views vie : Views.values()) {
        if (vie.equals(view)) {
            return true;
        }
    }
    return false;
}
\end{code}

\subsubsection{Test af endpoint klassen}
\begin{code}[caption=UsersEndpointTest.jar, firstnumber=8]
UsersEndpoint usersEndpoint;

public UsersEndpointTest() {
    usersEndpoint = new UsersEndpoint(new HttpManager());
}
\end{code}\\

UserEndpoint testen starter ved at den i dens constructer danner et nyt userEndpoint objekt. \\
postLogin testen tester om en der kan logges ind i systemet. Det gør den ved at hente den token man får tilbage når man logger ind. Det token er en string der er null hvis brugeren ikke kunne logge ind eller en lang string som giver brugen adgang til applikationen i sat stykke tid. \\
    

\begin{code}[caption=UsersEndpointTest.jar, firstnumber=34]
@Test
void postLogin() {
    var token = usersEndpoint.postLogin("string@string.dk", "string");
    assertNotEquals(null, token);
    token = usersEndpoint.postLogin("wronglogin@string.dk", "string");
    assertNull(token);
}
\end{code}    

Efter at have dannet token objektet tjekker den om at den ikke er null. Det gør den ved at bruge assertNotEquals metoden der tjekker om den værdi man giver den ikke er den forventede værdi. Så hvis token objektet ikker null består testen. \\

Herefter bliver der testet for det modsatte. Der bliver igen oprettet et token objekt med et forkert login så token objektet skulle gerne være null. Efter objektet er oprettet bliver det testet af assertNull der tester om værdien er null. Så hvis værdien er null består testen. \\

\subsubsection{Test af loginViewModel}
Testen af loginViewModel sker ved at der i constructeren bliver oprettet en loginViewModel\\
\begin{code}[caption=LoginViewModel.jar, firstnumber=16]
private final LoginViewModel loginViewModel;

/**
 * Instantiates a new Login view model test.
 */
public LoginViewModelTest() {
    var clientFactory = new ClientFactory();
    var modelFactory = new ModelFactory(clientFactory);
    var viewModelFactory = new ViewModelFactory(modelFactory);
    loginViewModel = viewModelFactory.getLoginViewModel();
}
\end{code}

Herefter bliver setUp metoden kørt før hver test.\\

\begin{code}[caption=LoginViewModel.jar, firstnumber=28]
@BeforeEach
void setUp() {
    loginViewModel.emailProperty().setValue("string@string.dk");
    loginViewModel.passwordProperty().setValue("string");
}
\end{code}

Her bliver loginViewModels email samt password sat.
Der er i alt fire test metoder, emailProperty, passwordProperty, loginResultProperty og clearFields.\\
    
\textbf{MailProperty} test metoden checker om den email fra setUp metoden er den samme som den forventede værdi.\\


\begin{code}[caption=LoginViewModel.jar, firstnumber=34]
@Test
void emailProperty() {
    assertEquals("string@string.dk", loginViewModel.emailProperty().get());
}
\end{code}

Det sker ved at kalde assertEquals metoden der tjekker om den forventede værdi er lig den givende værdi. I dette tilfælde tjekker den om emailen er \textbf{string@string.dk} og hvis den ikke er fejler testen.\\
    
\textbf{PasswordProperty} test metoden gør det samme som emailProperty test metoden, men i stedet for en email er det et password der bliver testet for.\\
\begin{code}[caption=LoginViewModel.jar, firstnumber=39]
@Test
void passwordProperty() {
    assertEquals("string", loginViewModel.passwordProperty().get());
}
\end{code}
    
\textbf{LoginResultProperty} test metoden bruger assertNull metoden, der fejler testen hvis det input den får er det samme som null. Her bliver loginViewModellens loginResponseProperty kaldt, for at se om den kan logge ind med de oplysninger som blev defineret ved setUp metoden. Så hvis den ikke kunne logge brugeren ind med de oplysninger den fik, returnerer den null og testen fejler.\\

\begin{code}[caption=LoginViewModel.jar, firstnumber=44]
@Test
void loginResultProperty() {
    assertNull(loginViewModel.loginResponseProperty().get());
}
\end{code}
    
\textbf{ClearFields} test metoden kører først loginViewModel’s clearFields metode, der fjerner tekst i email og passwordProperty. Herefter tjekker den med assertEqueals om de to er tomme, ved at sammenligne dem med et empty string. Så hvis email og password felterne er tomme består testen.\\

\begin{code}[caption=LoginViewModel.jar, firstnumber=49]
@Test
void clearFields() {
    loginViewModel.clearFields();
    assertEquals("", loginViewModel.emailProperty().get());
    assertEquals("", loginViewModel.passwordProperty().get());
}
\end{code}

\subsection{1. Iteration af api testning}
\subsubsection{Test af Users klassen}
UsersTest klassen importerer metoder fra klassen BestTestCase. Metoden test\_all er den der foretager testen på users ved at bruge metoderne i BestTestCase.\\

\begin{code}[language=python, caption=UsersTest.py, firstnumber=1]
from http import HTTPStatus

from tests.base_test import BaseTestCase

class UsersTest(BaseTestCase):
    path = "users"

    def test_all(self):
        self.get_list()

        response = self.post(json=self.json())
        identifier = response.json.get("identifier")
        self.patch(identifier=identifier, json=self.patch_json())
        self.get(identifier=identifier
        self.put(identifier=identifier, json=self.json())
        self.delete(identifier=identifier)
\end{code}

Der startes med at kalde get\_list på self, som er en metode importeret fra BestTestCase og ser således ud:\\

\begin{code}[language=python, caption=BestTestCase.py, firstnumber=55]
def get_list(self):
    response = self._get(path=f"/{self.path}/")
    self.assertTrue(response.status_code == HTTPStatus.OK)
    return response
\end{code}

Herefter bliver der lavet et response objekt, som bliver instantieret ved at kalde post metoden i BestTestCase med en parameter der bliver dannet i json metoden fra UsersTest klassen. \\

\begin{code}[language=python, caption=UsersTest.py, firstnumber=12]
response = self.post(json=self.json())
\end{code}

\begin{code}[language=python, caption=UsersTest.py, firstnumber=19]
def json(self, phone: str = None, email: str = None, name: str = None, password: str = None):
    password = password or self.random_string()
    return {
        "phone": phone or "+45 12 12 12 12",
        "email": email or f"{self.random_string()}@creditoro.nymann.dev",
        "name": name or self.random_string(),
        "password": password,
        "repeated_password": password
    }
\end{code}

Json har muligheden for at tage parametre, men behøver det ikke. Så hvis man ikke ønsker en specifik json får man en forudindstillet json. Det json der bliver oprettet har et telefonnummer, email, navn, password og repeated\_password.
Efter json er sat op kan det blive videregivet til post metoden i BestTestCase:\\

\begin{code}[language=python, caption=BestTestCase.py, firstnumber=65]
def post(self, data: dict = None, json: dict = None):
    response = self._post(path=f"/{self.path}/", data=data, json=json)
    self.assertTrue(response.status_code == HTTPStatus.CREATED)
    return response
\end{code}

Post metoden starter med at oprette et repsons objekt, ved at kalde metoden \_post med det json der blev videregivet. \\

\begin{code}[language=python, caption=BestTestCase.py, firstnumber=40]
def _post(self, path: str, data: dict = None, json: dict = None):
    return self.client.post(path, headers=self.headers, json=json, data=data)
\end{code}

I \_post metoden der kaldt en post metode på klienten, der returnerer et respons objekt.\\
I post metoden bliver respons objektet sat til det returnerede responsobjekt. Herefter bliver der testet om at HTTPStatus er CREATED ved brug af assertTrue der tjekker om response objektets status\_code er det samme som OK. Eller fejler testen. Og til sidst returnerer response objektet til test\_all metoden.\\

Efter at have oprettet response objektet, bliver der oprettet et identifier objekt. Dette objekt bliver sat til response objektets identifier værdi. Identifier bliver herefter brugt til at kalde patch metoden, der tager en identifier og værdien som patch\_json metoden returnere som parametre. \\

\begin{code}[language=python, caption=UsersTest.py, firstnumber=14]
self.patch(identifier=identifier, json=self.patch_json())
\end{code}

\begin{code}[language=python, caption=UserTest.py, firstnumber=29]
def patch_json(self, name: str = None):
    return {
        "name": name or self.random_string()
    }
\end{code}

Patch\_json ændrer enten navnet til noget bestemt eller tilfældigt, og returnerer det reviderede json objekt. De to parametre, identifier og json, bliver så videre givet til patch metoden i base\_test.\\

\begin{code}[language=python, caption=BestTestCase.py, firstnumber=70]
def patch(self, identifier: str, data: dict = None, json: dict = None):
    response = self._patch(path=f"/{self.path}/{identifier}", data=data, json=json)
    self.assertTrue(response.status_code == HTTPStatus.OK)
    return response
\end{code}

I patch metoden bliver der oprettet et nyt response objekt ved at kalde \_patch metoden. \_patch metoden returnere, ligeseom \_post metoden, et response objekt. Response objektet bliver igen testest igennem assertTrue metoden, så hvis responsens status\_code er OK, består testen.\\

Efter patch metoden bliver kaldt, bilver get metoden kaldt. Get metoden tager identifier objektet som parameter. \\

\begin{code}[language=python, caption=BestTestCase.py, firstnumber=60]
def get(self, identifier):
    response = self._get(path=f"/{self.path}/{identifier}")
    self.assertTrue(response.status_code == HTTPStatus.OK)
    return response
\end{code}

I get metoden bliver der, ligesom i patch metoden, oprettet et respons objekt. Her bliver det dog oprettet ved at kalde \_get metoden. Efter oprettelsen bliver den, ligesom de andre eksempler, testet for dens status\_code, der skal være \texttt{OK} for at testen lykkes.\\

Delete metoden bliver kaldt som den sidste i TetstUser test\_all metoden, og tager identifier objektet som paramter. Først bliver response objektet oprettet ved at kalde \_delete metoden, der returnere et response objekt. Herefter tester assertTrue metoden om reponse status\_code er lig med no\_content. De pågældene metoder kan ses nedenfor: \\

\begin{code}[language=python, caption=UsersTest.py, firstnumber=17]
self.delete(identifier=identifier)
\end{code}

\begin{code}[language=python, caption=BestCaseTest.py, firstnumber=80]
def delete(self, identifier: str):
    response = self._delete(f"/{self.path}/{identifier}")
    self.assertTrue(response.status_code == HTTPStatus.NO_CONTENT)
    return response
\end{code}
\newpage
    \newpage
    
    % Diskussion
    \section{Diskussion}
Denne sektion skal indeholde:

\begin{itemize}
    \item Hvad er der opnået og hvad er der ikke opnået i projektet i forhold til det forventede som beskrevet i indledningen.  
    \item Hvad er styrkerne og svaghederne ved jeres resultater.
    \item Kunne I have opnået bedre resultater.
\end{itemize}{}
    \newpage
    
    % Konklusion
    \section{Konklusion}

Det lykkedes gruppen at udvikle en prototype til et krediteringssystem, der giver mulighed for at erstatte rulletekster, efter endt TV-program. Med dette system kan man søge efter og finde krediteringer for programmer på forskellige kanaler, samt oprette nye krediteringer. Disse krediteringer skal være offentligt tilgængelige, og kunne redigeres og tilføjes af producere. Systemets opdeling af pakker følger design mønstret, MVVM, og gør det muligt at udskifte klienten til fx en website/webapplikation, da klienten kalder til et Rest API og gemmer i en database. Denne opdeling gør også at klasserne har en lav afhængighed. Allerede eksisterende data kan hentes via en EPG Poller fra TVTID.dk.
De indledende forventninger er at gruppen har kunnet lave en fuldt funktionel prototype, hvilket er blevet opnået. Enkelte funktionaliteter mangler dog i desktop-klienten. \myworries{Denne prototype er udviklet ved at følge Unified Process modellen og de tilhørende faser. Udformningen af kravspecifikationen og analysen viser hvilke krav og brugsmønstre systemet skal bygges op om, for at opfylde TV2's projektbeskrivelse.}\\
\myworries{INDSÆT NOGET OM DESIGN DELEN} \\
\myworries{INDSÆT NOGET OM TEST DELEN} \\
Gruppen har, ud fra de udarbejdede resultater i krav, analyse og design, udviklet prototypen til et krediteringssystem.\\

\\

\myworries{- Hvordan skal folk refereres = ????????} \\ %<--------------------------------------------
    \newpage
    
    % Perspektivering
    \section{Perspektivering}

Produktet gruppen har udviklet giver mulighed for at TV2 og andre kanaler kan flytte rullektekster fra TV og slutningen af programmer til en anden platform. Denne frigjorte plads kan derved udfyldes med reklamer og promoveringer for andre programmer. Eftersom systemet er bygget op omkring et Rest API, er det hurtigt og nemt at udskifte klienten til fx en website eller webapplikation. Derudover er systemet bygget op omkring muligheden for at tilføje flere kanaler end blot TV2, og er derfor en funktionel løsning på tværs af TV-stationer. Dette indebærer at oprette 'kanaler' til Boxer Play, Yousee, osv.\\
Løsningen gruppen er fundet frem til, har givet indsigt i hvorfor og hvordan man kan benytte et Rest API og EPG Poller. Dertil har brugen af design mønstret MVVM vist sig fordelagtigt til brug i større projekter med flere bidragsydere, grundet den lave afhængighed mellem klasser. \\

Til fremtidigt viderearbejdelse af projektet ville der først fokuseres på at færdiggøre prototypen, så den er fuldt funktionel, og indeholder alle brugsmønstre i MoSCoW analysen (se inceptionsdokumentet bilag \ref{inceptionsdokument}). Dette vil indebære at implementere brugerroller (system- og kanaladministrator, royal bruger, producer og gæst), så brugere tildelt en bestemt brugerrolle, vil have forskellige rettigheder i systemet, og derfor forskelligt indhold og muligheder. Derudover skal det være muligt at se personinformationer, godkende/afvise krediteringer og ændre sprog.
Systemet vil skulle samle alle episoder for en produktion under én titel, og inddele disse i sæsoner.
Det vil også give mening at lave en website/webapplikation til systemet. Fra denne website vil man også kunne finde og oprette krediteringer som i desktop-klienten. \\


    \newpage
    
    % Procesevaluering
    \section{Procesevaluering}
Denne sektion skal indeholde:

\begin{itemize}
    \item Processen og gruppens refleksion over processen:
        \item Læringsprocessen, teamroller, samarbejdet internt i gruppen og med vejleder, projektarbejdsformen, arbejdsformer, metoder, skriveprocessen, den tidsmæssige styring af projektet,ledelse af projektet, arbejdsfordeling i projektet m.m.
        \item Hvordan ville I gribe arbejdet an, hvis I skulle starte forfra.
\end{itemize}{}
    \newpage

    % bibliography empty
    \bibliographystyle{apalike}
    % \bibliographystyle{alphabetic}
    % \bibliographystyle{unsrt}
    \bibliography{appendices/bibliography.bib}
    % \medskip
    % \printbibliography
    \newpage
    
    % fjerner sidetal
    \pagenumbering{gobble}
    % Bilag
    \appendix % Bruges til bilag, så den laver bilag indholdfortegnelse og overskrifter
    \section{Bilag}

\subsection{Logbog}
\href{https://github.com/creditoro/logbook}{\textbf{Logbog} Github.com/creditoro/logbook}

\newpage
\section{Systemsekvensdiagrammer}

% ---------------------------- Opret kreditering ---------------------------------
\begin{figure}[H]
\centering
\includegraphics[scale=0.43]{figures/systemsekvensdiagrammer/opretKreditering.PNG}
\caption{Systemsekvensdiagram for "Opret kreditering"}
\label{fig:create_credit}
\end{figure}

% ---------------------------- Godkend kreditering -------------------------------
\begin{figure}[H]
\centering
\includegraphics[scale=0.43]{figures/systemsekvensdiagrammer/godkendKreditering.PNG}
\caption{Systemsekvensdiagram for "Godkend kreditering"}
\label{fig:approve_credit}
\end{figure}

% ---------------------------- Opret producer ------------------------------------
\begin{figure}[H]
\centering
\includegraphics[scale=0.43]{figures/systemsekvensdiagrammer/opretProducer.PNG}
\caption{Systemsekvensdiagram for "Opret producer"}
\label{fig:create_producer}
\end{figure}

% ---------------------------- Se personinformation ------------------------------
\begin{figure}[H]
\centering
\includegraphics[scale=0.43]{figures/systemsekvensdiagrammer/sePersonInfo.PNG}
\caption{Systemsekvensdiagram for "Se personinformation"}
\label{fig:read_person_info}
\end{figure}

\newpage
\input{appendices/operationssekvensdiagrammer}
\newpage

\end{document}