\documentclass[a4paper,11pt,danish,oneside]{article}
\usepackage[utf8]{inputenc}
\usepackage[danish]{babel}
\usepackage{graphicx}
\usepackage{multicol}
\usepackage[T1]{fontenc}
\usepackage[top=3cm, bottom=3cm, left=2.5cm, right=3.5cm]{geometry}
\usepackage{tocvsec2}
\usepackage{float}
\usepackage{tabularx}
\usepackage{parskip}
\usepackage{mathtools}
\usepackage{changepage}
\usepackage{mdwlist}
\usepackage{algorithm}
\usepackage{titlesec}
\usepackage{algpseudocode}
\usepackage{todonotes}
\algrenewcommand\algorithmicprocedure{\textbf{function}}
\usepackage[nottoc,numbib]{tocbibind}
\usepackage{setspace}
\usepackage[table]{xcolor}
\usepackage{tabularx}



\linespread{1.4}

\titlespacing*{\section}{0pt}{0.5cm}{0cm}
\titlespacing*{\subsection}{0pt}{0.3cm}{0cm}
\titlespacing*{\subsubsection}{0pt}{0.1cm}{0cm}
\titlespacing*{\paragraph}{0pt}{0cm}{0cm}
\graphicspath{ {./img/} }
\DeclareGraphicsExtensions{.pdf,.png,.jpg}
\begin{document}

\missingfigure

\section*{Section}
This is a section.
\subsection*{Subsection}
This is a sub-section.
\subsubsection*{subsubsection}
This is a subsub-section
\section{Section}
Section, again...
\subsection{Subsection}
And yet again :)
\subsubsection{subsubsection}
... And a little subsub-section... So cute.
{\setstretch{1}
    \begin{itemize}
    	\item First
    	\item Second
    	\item Third
    \end{itemize}
}

{\setstretch{1}
	\begin{enumerate}
		\item First
		\item Second
		\item Third
	\end{enumerate}
}

\begin{description}
	\setstretch{1}
	\item [First]\hfill \\  Description 
	\item [Second]\hfill \\  Description
	\item [Third]\hfill \\ Description
\end{description}

\begin{description}
	\setstretch{1}
	\item [First] -  Description 
	\item [Second] -  Description
	\item [Third] - Description
\end{description}

\begin{equation}
	s(n)=x(n)+2 \cos(2 \pi f) s(n-1)-s(n-2)
\end{equation}

This is an equation $s(n)=x(n)+2 \cos(2 \pi f) s(n-1)-s(n-2)$ it is in text.

\begin{table}[h]
	\begin{tabularx}{\textwidth}{|X|X|X|X|}
		\hline
		\multicolumn{2}{|l|}{\textbf{Goertzel}} & \multicolumn{2}{l|}{\textbf{FFT}} \\ \hline
	\textbf{Pro} & \textbf{Con} & \textbf{Pro} & \textbf{Con} \\ \hline
	First & Second & Third & Fourth \\ \hline
	& & & \\ \hline
	\end{tabularx}
	\caption{This is a tabel}
	\label{tabel1}
\end{table}

I'm referenceing a tabel l \ref{tabel1}

\begin{figure}[h]
	\centering
	\includegraphics[scale=0.2]{images/universe}
	\caption{Above is the file name, without extension}
	\label{monkeylars}
\end{figure}

Picture \ref{monkeylars} have the label monkeylars.


\begin{figure}[H]
	\centering
	\includegraphics[scale=0.2]{images/universe}
	\caption{ Placement of picture can be choosen after begin\{figure\}, 'H' is EXACTLY here, 'h' is roughly here, 't' is top, and 'b' is bottom.}
\end{figure}



\begin{algorithm}[h]
	\begin{algorithmic}[1]
		\Procedure{CRCencoder}{dataword}
		\State dividend = dataword
		\For{i from 0 to length of divisor - 1}
		\State append dividend with 0 
		\EndFor
		\For{i from 0 to length of dataword}
		\If{i'th element of dividend is 0}
		\For{j from 0 to length of divisor}
		\State set dividend i + j to dividend i + j \textbf{xor} with divisor j
		\EndFor
		\EndIf
		\EndFor
		\For{i from 0 to length of divisor - 1}
		\State append dataword with dividend i + length of dataword
		\EndFor
		\EndProcedure
	\end{algorithmic}
	\caption{Caption for the algoritm}
	\label{algo1}
\end{algorithm}

\begin{figure}[h]
	\begin{center}
		\begin{minipage}{0.45\textwidth}
			\centering
			\includegraphics[scale=1]{images/universe}
		\end{minipage}
		\begin{minipage}{0.45\textwidth}
			\centering
			\includegraphics[scale=1]{images/universe}
		\end{minipage}
		\caption{Here is two pictures, side by side with text}
	\end{center}
\end{figure}

\cite{Label}
Here we use the cite function, and the label from the bibliography. 
\cite[p.666]{Label}
It is also possible to cite with a page number.

\end{document}
