\section{Metode i elaborationsfasen}

%Fra Henrik: Indsæt beskrivelse af hvad der skal ske i Elaborationsfasen.

\subsection{UP \& Scrum}

Scrum vil blive brugt i elaborationsfasen, til at nedbryde de krav vi har defineret i inceptionsfasen (se krav). Der vil blive benyttet en sprint periode på 1 uge, da det liner op med det ugentlige vejledermøde. Da sprint perioden er kort (normalt bruges 1-4 uger) er det vigtigt at vi får brudt vores Epics ned til User Stories der kan nåes indenfor 1 sprint.
Vi vil i projektet bruge værktøjet ZenHub til GitHub for at integrere Scrum ind i projektet.
Dette giver os mulighed for at samle vores projekt management og kode inde på vores GitHub side (\url{https://github.com/creditoro}).

\noindent
Vi har til projektet et Kanban board med følgende kolonner:
\begin{itemize}
    \item New Issues
    \item Icebox
    \begin{itemize}
        \item Her er de issues som ikke er prioriteret eller er af lav prioritet (på nuværende tidspunkt).
    \end{itemize}
    \item Backlog
    \begin{itemize}
        \item Kommende issues der er prioriteret højt (sorteret fra top til bund på prioritet).
    \end{itemize}
    \item In Progress
    \begin{itemize}
        \item Hvad der bliver arbejdet på lige nu (sorteret fra top til bund på prioritet).
    \end{itemize}
    \item Done
    \begin{itemize}
        \item Issues der er færdige, disse vil blive lukket til næste sprint møde.
    \end{itemize}
    \item Closed
\end{itemize}
